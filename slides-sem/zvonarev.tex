\documentclass[unicode, notheorems]{beamer}

% If you have more than three sections or more than three subsections in at least one section,
% you might want to use the [compress] switch. In this case, only the current (sub-) section
% is displayed in the header and not the full overview.
\mode<presentation>
{
  \usetheme[numbers, totalnumbers, compress]{Statmod}

  \setbeamercovered{transparent}
  % or whatever (possibly just delete it)
}

%\usepackage{pscyr}
\usepackage[T2A]{fontenc}
\usepackage[utf8x]{inputenc}
\usepackage[russian]{babel}
\usepackage{amsmath,amssymb,amsthm,amscd,amsfonts}
\usepackage{mathdots}


%\usepackage{tikz}
% you only need this when using TikZ graphics

\newtheorem{theorem}{Теорема}
\newtheorem{example}{Пример}
\newtheorem{definition}{Определение}
\newtheorem{thnote}{Замечание}
\newtheorem{proposition}{Утверждение}
\newtheorem{solution}{Решение}

\DeclareMathOperator{\mathspan}{span}
\DeclareMathOperator{\rank}{rank}
\DeclareMathOperator{\cond}{cond}
\def\dist{\mathop{\mathrm{dist}}}

%new calligraphic font for subspaces
\usepackage{euscript}
\newcommand{\spA}{\EuScript{A}}
\newcommand{\spB}{\EuScript{B}}
\newcommand{\spC}{\EuScript{C}}
\newcommand{\spD}{\EuScript{D}}
\newcommand{\spE}{\EuScript{E}}
\newcommand{\spF}{\EuScript{F}}
\newcommand{\spG}{\EuScript{G}}
\newcommand{\spH}{\EuScript{H}}
\newcommand{\spI}{\EuScript{I}}
\newcommand{\spJ}{\EuScript{J}}
\newcommand{\spK}{\EuScript{K}}
\newcommand{\spL}{\EuScript{L}}
\newcommand{\spM}{\EuScript{M}}
\newcommand{\spN}{\EuScript{N}}
\newcommand{\spO}{\EuScript{O}}
\newcommand{\spP}{\EuScript{P}}
\newcommand{\spQ}{\EuScript{Q}}
\newcommand{\spR}{\EuScript{R}}
\newcommand{\spS}{\EuScript{S}}
\newcommand{\spT}{\EuScript{T}}
\newcommand{\spU}{\EuScript{U}}
\newcommand{\spV}{\EuScript{V}}
\newcommand{\spW}{\EuScript{W}}
\newcommand{\spX}{\EuScript{X}}
\newcommand{\spY}{\EuScript{Y}}
\newcommand{\spZ}{\EuScript{Z}}

%font for text indices like transposition X^\mathrm{T}
\newcommand{\rmA}{\mathrm{A}}
\newcommand{\rmB}{\mathrm{B}}
\newcommand{\rmC}{\mathrm{C}}
\newcommand{\rmD}{\mathrm{D}}
\newcommand{\rmE}{\mathrm{E}}
\newcommand{\rmF}{\mathrm{F}}
\newcommand{\rmG}{\mathrm{G}}
\newcommand{\rmH}{\mathrm{H}}
\newcommand{\rmI}{\mathrm{I}}
\newcommand{\rmJ}{\mathrm{J}}
\newcommand{\rmK}{\mathrm{K}}
\newcommand{\rmL}{\mathrm{L}}
\newcommand{\rmM}{\mathrm{M}}
\newcommand{\rmN}{\mathrm{N}}
\newcommand{\rmO}{\mathrm{O}}
\newcommand{\rmP}{\mathrm{P}}
\newcommand{\rmQ}{\mathrm{Q}}
\newcommand{\rmR}{\mathrm{R}}
\newcommand{\rmS}{\mathrm{S}}
\newcommand{\rmT}{\mathrm{T}}
\newcommand{\rmU}{\mathrm{U}}
\newcommand{\rmV}{\mathrm{V}}
\newcommand{\rmW}{\mathrm{W}}
\newcommand{\rmX}{\mathrm{X}}
\newcommand{\rmY}{\mathrm{Y}}
\newcommand{\rmZ}{\mathrm{Z}}

%tt font for time series
\newcommand{\tsA}{\mathbb{A}}
\newcommand{\tsB}{\mathbb{B}}
\newcommand{\tsC}{\mathbb{C}}
\newcommand{\tsD}{\mathbb{D}}
\newcommand{\tsE}{\mathbb{E}}
\newcommand{\tsF}{\mathbb{F}}
\newcommand{\tsG}{\mathbb{G}}
\newcommand{\tsH}{\mathbb{H}}
\newcommand{\tsI}{\mathbb{I}}
\newcommand{\tsJ}{\mathbb{J}}
\newcommand{\tsK}{\mathbb{K}}
\newcommand{\tsL}{\mathbb{L}}
\newcommand{\tsM}{\mathbb{M}}
\newcommand{\tsN}{\mathbb{N}}
\newcommand{\tsO}{\mathbb{O}}
\newcommand{\tsP}{\mathbb{P}}
\newcommand{\tsQ}{\mathbb{Q}}
\newcommand{\tsR}{\mathbb{R}}
\newcommand{\tsS}{\mathbb{S}}
\newcommand{\tsT}{\mathbb{T}}
\newcommand{\tsU}{\mathbb{U}}
\newcommand{\tsV}{\mathbb{V}}
\newcommand{\tsW}{\mathbb{W}}
\newcommand{\tsX}{\mathbb{X}}
\newcommand{\tsY}{\mathbb{Y}}
\newcommand{\tsZ}{\mathbb{Z}}

%bf font for matrices
\newcommand{\bfA}{\mathbf{A}}
\newcommand{\bfB}{\mathbf{B}}
\newcommand{\bfC}{\mathbf{C}}
\newcommand{\bfD}{\mathbf{D}}
\newcommand{\bfE}{\mathbf{E}}
\newcommand{\bfF}{\mathbf{F}}
\newcommand{\bfG}{\mathbf{G}}
\newcommand{\bfH}{\mathbf{H}}
\newcommand{\bfI}{\mathbf{I}}
\newcommand{\bfJ}{\mathbf{J}}
\newcommand{\bfK}{\mathbf{K}}
\newcommand{\bfL}{\mathbf{L}}
\newcommand{\bfM}{\mathbf{M}}
\newcommand{\bfN}{\mathbf{N}}
\newcommand{\bfO}{\mathbf{O}}
\newcommand{\bfP}{\mathbf{P}}
\newcommand{\bfQ}{\mathbf{Q}}
\newcommand{\bfR}{\mathbf{R}}
\newcommand{\bfS}{\mathbf{S}}
\newcommand{\bfT}{\mathbf{T}}
\newcommand{\bfU}{\mathbf{U}}
\newcommand{\bfV}{\mathbf{V}}
\newcommand{\bfW}{\mathbf{W}}
\newcommand{\bfX}{\mathbf{X}}
\newcommand{\bfY}{\mathbf{Y}}
\newcommand{\bfZ}{\mathbf{Z}}

%bb font for standard spaces and expectation
\newcommand{\bbA}{\mathbb{A}}
\newcommand{\bbB}{\mathbb{B}}
\newcommand{\bbC}{\mathbb{C}}
\newcommand{\bbD}{\mathbb{D}}
\newcommand{\bbE}{\mathbb{E}}
\newcommand{\bbF}{\mathbb{F}}
\newcommand{\bbG}{\mathbb{G}}
\newcommand{\bbH}{\mathbb{H}}
\newcommand{\bbI}{\mathbb{I}}
\newcommand{\bbJ}{\mathbb{J}}
\newcommand{\bbK}{\mathbb{K}}
\newcommand{\bbL}{\mathbb{L}}
\newcommand{\bbM}{\mathbb{M}}
\newcommand{\bbN}{\mathbb{N}}
\newcommand{\bbO}{\mathbb{O}}
\newcommand{\bbP}{\mathbb{P}}
\newcommand{\bbQ}{\mathbb{Q}}
\newcommand{\bbR}{\mathbb{R}}
\newcommand{\bbS}{\mathbb{S}}
\newcommand{\bbT}{\mathbb{T}}
\newcommand{\bbU}{\mathbb{U}}
\newcommand{\bbV}{\mathbb{V}}
\newcommand{\bbW}{\mathbb{W}}
\newcommand{\bbX}{\mathbb{X}}
\newcommand{\bbY}{\mathbb{Y}}
\newcommand{\bbZ}{\mathbb{Z}}

%got font for any case
\newcommand{\gA}{\mathfrak{A}}
\newcommand{\gB}{\mathfrak{B}}
\newcommand{\gC}{\mathfrak{C}}
\newcommand{\gD}{\mathfrak{D}}
\newcommand{\gE}{\mathfrak{E}}
\newcommand{\gF}{\mathfrak{F}}
\newcommand{\gG}{\mathfrak{G}}
\newcommand{\gH}{\mathfrak{H}}
\newcommand{\gI}{\mathfrak{I}}
\newcommand{\gJ}{\mathfrak{J}}
\newcommand{\gK}{\mathfrak{K}}
\newcommand{\gL}{\mathfrak{L}}
\newcommand{\gM}{\mathfrak{M}}
\newcommand{\gN}{\mathfrak{N}}
\newcommand{\gO}{\mathfrak{O}}
\newcommand{\gP}{\mathfrak{P}}
\newcommand{\gQ}{\mathfrak{Q}}
\newcommand{\gR}{\mathfrak{R}}
\newcommand{\gS}{\mathfrak{S}}
\newcommand{\gT}{\mathfrak{T}}
\newcommand{\gU}{\mathfrak{U}}
\newcommand{\gV}{\mathfrak{V}}
\newcommand{\gW}{\mathfrak{W}}
\newcommand{\gX}{\mathfrak{X}}
\newcommand{\gY}{\mathfrak{Y}}
\newcommand{\gZ}{\mathfrak{Z}}

%old calligraphic font
\newcommand{\calA}{\mathcal{A}}
\newcommand{\calB}{\mathcal{B}}
\newcommand{\calC}{\mathcal{C}}
\newcommand{\calD}{\mathcal{D}}
\newcommand{\calE}{\mathcal{E}}
\newcommand{\calF}{\mathcal{F}}
\newcommand{\calG}{\mathcal{G}}
\newcommand{\calH}{\mathcal{H}}
\newcommand{\calI}{\mathcal{I}}
\newcommand{\calJ}{\mathcal{J}}
\newcommand{\calK}{\mathcal{K}}
\newcommand{\calL}{\mathcal{L}}
\newcommand{\calM}{\mathcal{M}}
\newcommand{\calN}{\mathcal{N}}
\newcommand{\calO}{\mathcal{O}}
\newcommand{\calP}{\mathcal{P}}
\newcommand{\calQ}{\mathcal{Q}}
\newcommand{\calR}{\mathcal{R}}
\newcommand{\calS}{\mathcal{S}}
\newcommand{\calT}{\mathcal{T}}
\newcommand{\calU}{\mathcal{U}}
\newcommand{\calV}{\mathcal{V}}
\newcommand{\calW}{\mathcal{W}}
\newcommand{\calX}{\mathcal{X}}
\newcommand{\calY}{\mathcal{Y}}
\newcommand{\calZ}{\mathcal{Z}}

%sf font for transposition and spaces like R
\newcommand{\sfA}{\mathsf{A}}
\newcommand{\sfB}{\mathsf{B}}
\newcommand{\sfC}{\mathsf{C}}
\newcommand{\sfD}{\mathsf{D}}
\newcommand{\sfE}{\mathsf{E}}
\newcommand{\sfF}{\mathsf{F}}
\newcommand{\sfG}{\mathsf{G}}
\newcommand{\sfH}{\mathsf{H}}
\newcommand{\sfI}{\mathsf{I}}
\newcommand{\sfJ}{\mathsf{J}}
\newcommand{\sfK}{\mathsf{K}}
\newcommand{\sfL}{\mathsf{L}}
\newcommand{\sfM}{\mathsf{M}}
\newcommand{\sfN}{\mathsf{N}}
\newcommand{\sfO}{\mathsf{O}}
\newcommand{\sfP}{\mathsf{P}}
\newcommand{\sfQ}{\mathsf{Q}}
\newcommand{\sfR}{\mathsf{R}}
\newcommand{\sfS}{\mathsf{S}}
\newcommand{\sfT}{\mathsf{T}}
\newcommand{\sfU}{\mathsf{U}}
\newcommand{\sfV}{\mathsf{V}}
\newcommand{\sfW}{\mathsf{W}}
\newcommand{\sfX}{\mathsf{X}}
\newcommand{\sfY}{\mathsf{Y}}
\newcommand{\sfZ}{\mathsf{Z}}

\title[Аппроксимация рядами конечного ранга]{Аппроксимация временных рядов рядами конечного ранга}

\author[Звонарев Никита]{Звонарев Никита}
\institute[СПбГУ]{Санкт-Петербургский государственный университет \\
    Кафедра статистического моделирования
}
\date{
    Санкт-Петербург\\
    2015г.
}

\subject{Beamer}
\setbeamertemplate{bibliography item}{[\theenumiv]}
% This is only inserted into the PDF information catalog. Can be left
% out.

% Delete this, if you do not want the table of contents to pop up at
% the beginning of each subsection:
% \AtBeginSubsection[]
% {
%   \begin{frame}<beamer>
%     \frametitle{Outline}
%     \tableofcontents[currentsection,currentsubsection]
%   \end{frame}
% }

\begin{document}

\begin{frame}
    \titlepage
\end{frame}


\section{Введение}
\begin{frame}
	\frametitle{Постановка задачи}
	Дан временной ряд $\tsX = \tsS + \tsN$, $\tsX = (x_1, \ldots, x_N)$, 
	
	$\tsN$ --- белый шум.

	$\tsS$ --- сигнал, имеющий \textit{определённую структуру}, например:
	\begin{enumerate}
		\item $\tsS$ удовлетворяет линейной рекуррентной формуле ($s_n = a_1 s_{n-1} + a_2 s_{n-2}$)
		\item $\tsS$ удовлетворяет параметрической модели ($s_n = C e^{\alpha n} \cos(2 \pi \omega n + \psi)$ или $s_n = C_1 e^{\alpha_1 n} + C_2 e^{\alpha_2 n}$)
	\end{enumerate}
	
	\textcolor{blue}{Задача:} построить оценку $\hat \tsS$ сигнала $\tsS$.
	
	\pause
	
	\textcolor{blue}{Выбираем первый подход}
\end{frame}

\begin{frame}
	\frametitle{Модель сигнала}
	Общий вид:
	
	\begin{definition}
		Ряд, управляемый минимальной линейной рекуррентной формулой порядка~$r$: $\tsS = (s_1, \ldots, s_N), \quad s_n = \sum_{i = 1}^{r} a_i s_{n-i}, \, n = r + 1, \ldots, N, \, a_r \ne 0$.
	\end{definition}
	
	\vspace{0.5cm}
	Параметрический вид рядов, управляемых ЛРФ:
	\begin{equation*}
	s_n = \sum_i P_i(n) \exp(\alpha_i n) \cos(2 \pi \omega_i n + \psi_i).
	\end{equation*}
	
	Параметрический подход не работает из-за неустойчивости
	
\end{frame}


\begin{frame}
	\frametitle{Идея: траекторная матрица}
	Возьмём $\tsS = (s_1, \ldots, s_N)$ --- ряд, управляемый минимальной ЛРФ порядка $r$: $ s_n = \sum_{i = 1}^{r} a_i s_{n-i}$.
	
	$1 < L < N$ --- \emph{длина окна}, $K = N - L + 1$.
	
	\emph{Траекторная матрица ряда} $\tsS$:
	
	\begin{equation*}
	\bfS = \calT(\tsS) = \begin{pmatrix}
	s_1 & s_2 & \ldots & s_K \\
	s_2 & s_3 & \ldots & s_{K + 1} \\
	\vdots & \vdots & \vdots & \vdots \\
	s_L & s_{L + 1} & \ldots & s_N
	\end{pmatrix}.
	\end{equation*}
	
	Заметим, что $\bfS$ --- ганкелева (\textit{structured}, $\bfS \in \calH$) и \\ $\rank \bfS = r$ (\textit{low-rank}, $\bfS \in \calM_r$). 
	
	$\tsX = \tsS + \tsN$, $\bfX = \calT(\tsX) = \bfS + \bfN$
	
	Оценка $\hat \tsS$ $\Leftrightarrow$ \textbf{SLRA} (Structured (Hankel) Low Rank Approximation) матрицы $\bfX$ в классе $\calH \cap \calM_r$.
	% Если получим $\hat \bfS$ --- оценку $\bfS$, то $\hat \tsS = \calT^{-1}(\hat \bfS)$ --- оценка сигнала.
	
	
%	Заметим, что $\bfS$ --- ганкелева ($\bfS \in \calH$) и $\text{rank} \bfS = r$.
	
%	\begin{definition}
%	    Ряд $\tsS$ имеет L-ранг $r$ если $\operatorname{rank} \bfS = r$, где $\bfS = \calT(\tsS)$.  
%	\end{definition}
	
%	Ряд $\tsS$ управляется минимальной ЛРФ порядка $r$ $\Rightarrow$ $\tsS$ имеет L-ранг $r$.   
	
%	$\sfX^r_N = \{\tsX : \tsX = (x_1, \ldots, x_N), \; \tsX \; \text{имеет L-ранг} \; \le r\}$
	
\end{frame}

\begin{frame}
	\frametitle{Эквивалентные задачи}
	\vspace{-0.2cm}
	$\tsX = (x_1, \ldots, x_N)$, $\bfX = \calT(\tsX)$. 
	
	Задача аппроксимации матрицы (SLRA):
	\begin{equation*} \label{matrixtask}
	\|\bfX - \bfY\|^2_F \to \min_{\substack{\bfY \in \calM_r \cap \calH}},
	\end{equation*}
	%$\|\cdot \|_M$ порождена следующим скалярным произведением: $\langle\bfX, \bfY \rangle_M = \sum_{l = 1}^L \sum_{k = 1}^K m_{l, k} x_{l, k} y_{l, k}$
	\vspace{0.2cm}
	Эквивалентная задача аппроксимации рядов:
	\begin{equation*}
	\sum \limits_{i=1}^N w_i(x_i - y_i)^2 \to \min_{\tsY \in \calX_N^r},
	\end{equation*}
	$\calX_N^r$ --- множество рядов, управляемых ЛРФ,
	\begin{equation*}
	w_i = \begin{cases}
	i & \text{для $i = 1, \ldots, L-1,$}\\
	L & \text{для $i = L, \ldots, K,$}\\
	N - i + 1 & \text{для $i = K + 1, \ldots, N,$}
	\end{cases} \qquad \text{---}
	\end{equation*}
	трапециевидные веса.
\end{frame}

\begin{frame}
	\frametitle{Методы Cadzow и SSA}
	
	SLRA:
	\begin{equation*}
		\|\bfX - \bfY\|^2_F \to \min_{\substack{\bfY \in \calM_r \cap \calH}}
	\end{equation*}

	Решение --- \textcolor{blue}{метод Cadzow (1988)}: 
	\begin{equation*}
	\bfY_k = (\Pi_\calH \circ \Pi_{\calM_r})^k (\bfX),
	\end{equation*}
	$\Pi_\calH$ --- проектор на ганкелевы матрицы, $\Pi_{\calM_r}$ --- проектор на матрицы ранга $\le r$, $k$ --- число итераций.
			

	$k = 1$ --- метод SSA (Singular Spectrum Analysis, ссылки). Не дает оценку из $\calM_r \cap \calH$.
	
	\vspace{0.4cm}\pause		
	Что известно про Cadzow?
\end{frame}

\begin{frame}
	\frametitle{Метод Cadzow}
	\begin{equation*}
	\bfY_k = (\Pi_\calH \circ \Pi_{\calM_r})^k (\bfX).
	\end{equation*}
	
	Ничего не известно ни про сходимость к глобальному, ни к локальному минимуму (понятие ``локального экстремума'' отсутствует, т.к. его не ввести естественным образом)

	\textcolor{blue}{Но:} 

	\begin{theorem}[Cadzow, 1988]
		\begin{enumerate}
			\item $\|\bfY_k - \Pi_{\calM_r} \bfY_k\| \to 0$ при $k \to +\infty$, $\|\Pi_{\calM_r}\bfY_k - \bfY_{k+1}\| \to 0$ при $k \to +\infty$.
			\item Существует сходящаяся подпоследовательность точек $\bfY_{i_1}, \bfY_{i_2}, \ldots$ такая, что ее предел $\bfY^*$  лежит в $\calM_r \cap \calH$.
		\end{enumerate}
	\end{theorem}
\end{frame}

\section{Выбор норм, соответствие между весами}
\begin{frame}
\frametitle{Напоминание: эквивалентные задачи}
Аппроксимация рядов:
\begin{equation*}
\sum \limits_{i=1}^N w_i(x_i - y_i)^2 \to \min_{\tsY \in \calX_N^r},
\end{equation*}
\begin{equation*}
w_i = \begin{cases}
i & \text{для $i = 1, \ldots, L-1,$}\\
L & \text{для $i = L, \ldots, K,$}\\
N - i + 1 & \text{для $i = K + 1, \ldots, N$}
\end{cases}
\end{equation*}

	\vspace{0.4cm}
\textit{Наша задача --- с помощью аппроксимации ряда оценить сигнал.}

	\vspace{0.4cm}
Можно ли обобщить Cadzow и использовать другие веса вместо $w_i$?
\end{frame}

\begin{frame}
	\frametitle{Варианты (полу)норм}
	Ряды:
	$\|\tsX\|_q^2 = \sum_{i = 1}^N q_i x_i^2$, все $q_i \ge 0$. \textit{Но как аппроксимировать по }$\|\tsX\|_q$?
	
	\vspace{0.4cm}
	Решение: подобрать эквивалентную матричную норму
	
	\begin{enumerate}
		\item $\|\bfX\|^2_{\bfD_1, \bfD_2} = \text{tr}(\bfD_1 \bfX \bfD_2 \bfX^\rmT)$, $\bfD_1 \in \sfR^{L \times L}$, $\bfD_2 \in \sfR^{K \times K}$ --- симметричные, неотрицательно определённые матрицы (+ диагональные в нашем случае) (Zhigljavsky, 2014)
		\item Частный случай предыдущего пункта: $\bfD_1$, $\bfD_2$ --- единичные матрицы (Фробениусовская норма $\|\bfX\|_F$)

	\end{enumerate}
	
	\vspace{0.4cm}
	Проблемы:
	\begin{enumerate}
		\item Как вычислить $\Pi_\calH$, $\Pi_{\calM_r}$ по нужной норме?
		\item Как подобрать $\bfD_1$ и $\bfD_2$, соответствующие $q_i$: $\|\tsX\|_q = \|\bfX\|_{\bfD_1, \bfD_2}$?
	\end{enumerate}
\end{frame}

\begin{frame}
	\frametitle{Проекторы $\Pi_{\calM_r}$ и $\Pi_{\calH}$}
	
	$\Pi_{\calM_r}$: через косоугольное SVD-разложение $\bfX$. Сводится к стандартному SVD-разложению матрицы $\bfO_1 \bfX \bfO_2^\rmT$, $\bfO_i^\rmT \bfO_i = \bfD_i$, $i = 1, 2.$
		
	\vspace{0.6cm}
	$\Pi_{\calH}$: через взвешенное диагональное усреднение
	\begin{equation*}
	\bfX = \Pi_{\calH}(\bfY), \quad x_{l,k} = \frac{\sum_{i,j:\, i+j=l+k} d_{1, i} d_{2, j} y_{i,j}}{\sum_{i,j:\, i+j=l+k} d_{1, i} d_{2, j}}.
	\end{equation*}
\end{frame}


\begin{frame}
	\frametitle{Эквивалентность и равенство норм}
    $\|\tsX\|_q^2 = \sum_{i = 1}^N q_i x_i^2$, все $q_i > 0$
    
    $\|\bfX\|^2_{\bfD_1, \bfD_2} = \text{tr}(\bfD_1 \bfX \bfD_2 \bfX^\rmT)$
    
    $\bfD_1 = \text{diag}(d_{1, 1}, \ldots, d_{1, L})$, $\bfD_2 = \text{diag}(d_{2, 1}, \ldots, d_{2, K})$.
	
	\begin{theorem}
		Полунорма $\|\cdot\|_{\bfD_1, \bfD_2}$ является нормой $\Leftrightarrow$ все $d_{1, i} > 0$, $i = 1, \ldots, L$, $d_{2, j} > 0$, $j = 1, \ldots, K$.
	\end{theorem}
	\vspace{0.2cm}
	Векторы $D_1$, $D_2$, $Q \in \sfR^N$, $D_1 = (d_{1, 1}, \ldots, d_{1, L}, 0, \ldots, 0)^\rmT$, $D_2 = (d_{2, 1}, \ldots, d_{2, K}, 0, \ldots, 0)^\rmT$, $Q = (q_1, \ldots, q_N)^\rmT$.
	
	Свертка: $C = A \star B$, $c_{i} = \sum_{j=0}^{N-1} a_{j} b_{i-j}$, $A = (a_0, \ldots, a_{N-1})^\rmT$ \ldots
	\begin{theorem}[Zhigljavsky, 2015]
		$\|\calT(\tsX)\|_{\bfD_1, \bfD_2} = \|\tsX\|_q$ для любого $\tsX$ $\Leftrightarrow$ $Q = D_1 \star D_2$.
	\end{theorem}
\end{frame}

\begin{frame}
	\frametitle{Единичные веса ряда}
	\vspace{-0.3cm}
	\begin{equation*}
	\sum \limits_{i=1}^N q_i(x_i - y_i)^2 \to \min_{\tsY \in \calX_N^r}
	\end{equation*}
	
    \textcolor{blue}{Естественные веса:} $q_i = 1$.
	
	\vspace{0.3cm}
	
	Вопрос: \textcolor{blue}{как найти невырожденные $\bfD_1$, $\bfD_2$ такие, что $D_1 \star D_2 = (1, \ldots, 1)^\rmT$?}
	
	Anatoly Zhigljavsky, 2015: $D_1$ и $D_2$ могут состоять только из нулей и единиц $\Rightarrow$ \textcolor{red}{таких невырожденных $D_1$ и $D_2$ не существует,} значит, невозможно найти эквивалентную матричную задачу.
	
	\vspace{0.4cm}
	Пример: $N = 12$, $L = 4$, $K = 9$. $\bfD_1 = \text{diag}(1, 1, 1, 1)$, 
	$\bfD_2 = \text{diag}(1, 0, 0, 0, 1, 0, 0, 0, 1)$.
\end{frame}

\begin{frame}
	\frametitle{Поиск весов}
	\begin{equation*}
	Q = D_1 \star D_2.
	\end{equation*}
	
	\textcolor{blue}{Идея:} потерять точность равенства, но избавиться от вырожденности. 
	
	Зафиксируем $\bfD_1$ --- единичная матрица. Аппроксимация требуемых единичных весов: зафиксируем параметр $0 < \alpha < 1$, минимизация по диагональной положительно определённой $\bfD_2$:
	\begin{equation*}
	\|D_1 \star D_2 - (1, \ldots, 1)^\rmT\|_\cdot \to \min_{\cond \bfD_2 \le \frac{1}{\alpha}}
	\end{equation*}
	
	\vspace{0.2cm}
	Condition number (число обусловленности) $\cond \bfD = \|D\| \|D\|^{-1} = $ $\frac{\max_i d_i}{\min_i d_i}$, если $\bfD = \text{diag}(d_1, \ldots, d_N)$.
\end{frame}

\begin{frame}
	\frametitle{План}
	\begin{equation}\label{plan_approx}
	\|D_1 \star D_2 - (1, \ldots, 1)^\rmT\|_\cdot \to \min_{\cond \bfD_2 \le \frac{1}{\alpha}}
	\end{equation}
	
	\begin{enumerate}
		\item Как решить задачу \eqref{plan_approx}?
		Если $\|\cdot\|_\cdot = \|\cdot\|_1$ или $\|\cdot\|_\infty$ --- сводится к задаче линейного программирования
		
		$\|\cdot\|_\cdot = \|\cdot\|_2$ --- квадратичное программирование, есть очень быстрое решение при $L \approx N/2$ (``Active set'' метод, оптимальный выбор начальной точки)
		\pause
		\item Как решить задачу \eqref{plan_approx}, если не фиксировать $\bfD_1$?
		
		Открытый вопрос: невыпуклая целевая функция (приближенное решение --- эвристика), неясные ограничения.
		\pause
		\item Какой выигрыш дают примерно единичные веса?
	\end{enumerate}
	
\end{frame}

\section{Численные эксперименты}
\begin{frame}
	\frametitle{Сравнение (зависимость RMSE от числа итераций)}
	Задача оценки сигнала: $N = 40$, $L = 20$, $r = 2$, $\tsX = (x_{1}, \ldots, x_N),$  $x_k = 5\sin{\frac{2 k \pi}{6}} + \varepsilon_k$; $\varepsilon_k$, $k = 1, \ldots N$, --- гауссовский белый шум, $\tsE \varepsilon_k = 0$, $\tsD \varepsilon_k = 1$, 1000 реализаций.
	
	\vspace{-1cm}
	\begin{center}
		\includegraphics*[width = 10cm]{threealphas.pdf}
	\end{center}
\end{frame}

\begin{frame}
	\frametitle{Сравнение по трудоемкости}
	Задача оценки сигнала: $N = 40$, $L = 20$, $r = 2$, $\tsX = (x_{1}, \ldots, x_N),$  $x_k = 5\sin{\frac{2 k \pi}{6}} + \varepsilon_k$.
	
	Критерий остановки: $\frac{\|\calT^{-1}(\bfY_k) - \calT^{-1}(\bfY_{k + 1})\|^2}{N} < 10^{-8}$
	\vspace{-0.4cm}
	\begin{center}
		\includegraphics*[width = 9cm]{2axis.pdf}
	\end{center}
\end{frame}

\begin{frame}
	\frametitle{Чередование проекций при различном параметре $\alpha$}
	\vspace{-0.2cm}
	\textcolor{blue}{Идея:} совместить малые и большие $\alpha$, чтобы получить и быструю, и точную оценку.
	
	Целевая функция имеет большое число ``локальных минимумов''.
	
	\vspace{0.2cm} При большом $\alpha$ ($\alpha = 1$): быстрая сходимость, но далеко от оптимальной окрестности
	
	\vspace{0.2cm} При малом $\alpha$: медленная сходимость, но близко к оптимальной окрестности
    \begin{center}
		\includegraphics*[width = 7.5cm]{optim.pdf} 
    \end{center}
\end{frame}

\begin{frame}
	\frametitle{Сравнение (RMSE при чередовании $\alpha$)}
	Задача оценки сигнала: $N = 40$, $L = 20$, $r = 2$, $\tsX = (x_{1}, \ldots, x_N),$  $x_k = 5\sin{\frac{2 k \pi}{6}} + \varepsilon_k$; $\varepsilon_k$, $k = 1, \ldots N$, --- гауссовский белый шум, $\tsE \varepsilon_k = 0$, $\tsD \varepsilon_k = 1$, 1000 реализаций.
	
	\vspace{-1cm}
	\begin{center}
		\includegraphics*[width = 10cm]{various.pdf}
	\end{center}
\end{frame}

\section{Итоги}

\begin{frame}
	\frametitle{Планы на будущее}
	\begin{itemize}
		\item Теоретически установить соотношение между скоростью сходимости и разделимостью
		\item Оценить разделимость теоретически через $\cond \bfD_2$ в общем случае
		
		(разделимость $\leftrightarrow$ $\alpha$: есть результат для весов определённого вида при $N \vdots L$ (Zvonarev, 2015))
		\item Использование двусторонних весов (без фиксирования $\bfD_1$).
		
		Включает в себя два предыдущих вопроса
		\item Использование недиагональных $\bfD_1$ и $\bfD_2$ (например, для красного шума)
	\end{itemize}
	
\end{frame}

\bibliographystyle{plain}

\end{document}