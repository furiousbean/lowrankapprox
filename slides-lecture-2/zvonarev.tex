\documentclass[unicode, notheorems]{beamer}

% If you have more than three sections or more than three subsections in at least one section,
% you might want to use the [compress] switch. In this case, only the current (sub-) section
% is displayed in the header and not the full overview.
\mode<presentation>
{
  \usetheme[numbers, totalnumbers, compress]{Statmod}

  \setbeamercovered{transparent}
  % or whatever (possibly just delete it)
}

%\usepackage{pscyr}
\usepackage[T2A]{fontenc}
\usepackage[utf8x]{inputenc}
\usepackage[russian]{babel}
\usepackage{amsthm}
\usepackage{mathdots}


%\usepackage{tikz}
% you only need this when using TikZ graphics

\newtheorem{theorem}{Теорема}
\newtheorem{example}{Пример}
\newtheorem{definition}{Определение}
\newtheorem{thnote}{Замечание}
\newtheorem{proposition}{Утверждение}
\newtheorem{solution}{Решение}

\DeclareMathOperator{\mathspan}{span}
\def\dist{\mathop{\mathrm{dist}}}

%new calligraphic font for subspaces
\usepackage{euscript}
\newcommand{\spA}{\EuScript{A}}
\newcommand{\spB}{\EuScript{B}}
\newcommand{\spC}{\EuScript{C}}
\newcommand{\spD}{\EuScript{D}}
\newcommand{\spE}{\EuScript{E}}
\newcommand{\spF}{\EuScript{F}}
\newcommand{\spG}{\EuScript{G}}
\newcommand{\spH}{\EuScript{H}}
\newcommand{\spI}{\EuScript{I}}
\newcommand{\spJ}{\EuScript{J}}
\newcommand{\spK}{\EuScript{K}}
\newcommand{\spL}{\EuScript{L}}
\newcommand{\spM}{\EuScript{M}}
\newcommand{\spN}{\EuScript{N}}
\newcommand{\spO}{\EuScript{O}}
\newcommand{\spP}{\EuScript{P}}
\newcommand{\spQ}{\EuScript{Q}}
\newcommand{\spR}{\EuScript{R}}
\newcommand{\spS}{\EuScript{S}}
\newcommand{\spT}{\EuScript{T}}
\newcommand{\spU}{\EuScript{U}}
\newcommand{\spV}{\EuScript{V}}
\newcommand{\spW}{\EuScript{W}}
\newcommand{\spX}{\EuScript{X}}
\newcommand{\spY}{\EuScript{Y}}
\newcommand{\spZ}{\EuScript{Z}}

%font for text indices like transposition X^\mathrm{T}
\newcommand{\rmA}{\mathrm{A}}
\newcommand{\rmB}{\mathrm{B}}
\newcommand{\rmC}{\mathrm{C}}
\newcommand{\rmD}{\mathrm{D}}
\newcommand{\rmE}{\mathrm{E}}
\newcommand{\rmF}{\mathrm{F}}
\newcommand{\rmG}{\mathrm{G}}
\newcommand{\rmH}{\mathrm{H}}
\newcommand{\rmI}{\mathrm{I}}
\newcommand{\rmJ}{\mathrm{J}}
\newcommand{\rmK}{\mathrm{K}}
\newcommand{\rmL}{\mathrm{L}}
\newcommand{\rmM}{\mathrm{M}}
\newcommand{\rmN}{\mathrm{N}}
\newcommand{\rmO}{\mathrm{O}}
\newcommand{\rmP}{\mathrm{P}}
\newcommand{\rmQ}{\mathrm{Q}}
\newcommand{\rmR}{\mathrm{R}}
\newcommand{\rmS}{\mathrm{S}}
\newcommand{\rmT}{\mathrm{T}}
\newcommand{\rmU}{\mathrm{U}}
\newcommand{\rmV}{\mathrm{V}}
\newcommand{\rmW}{\mathrm{W}}
\newcommand{\rmX}{\mathrm{X}}
\newcommand{\rmY}{\mathrm{Y}}
\newcommand{\rmZ}{\mathrm{Z}}

%tt font for time series
\newcommand{\tsA}{\mathbb{A}}
\newcommand{\tsB}{\mathbb{B}}
\newcommand{\tsC}{\mathbb{C}}
\newcommand{\tsD}{\mathbb{D}}
\newcommand{\tsE}{\mathbb{E}}
\newcommand{\tsF}{\mathbb{F}}
\newcommand{\tsG}{\mathbb{G}}
\newcommand{\tsH}{\mathbb{H}}
\newcommand{\tsI}{\mathbb{I}}
\newcommand{\tsJ}{\mathbb{J}}
\newcommand{\tsK}{\mathbb{K}}
\newcommand{\tsL}{\mathbb{L}}
\newcommand{\tsM}{\mathbb{M}}
\newcommand{\tsN}{\mathbb{N}}
\newcommand{\tsO}{\mathbb{O}}
\newcommand{\tsP}{\mathbb{P}}
\newcommand{\tsQ}{\mathbb{Q}}
\newcommand{\tsR}{\mathbb{R}}
\newcommand{\tsS}{\mathbb{S}}
\newcommand{\tsT}{\mathbb{T}}
\newcommand{\tsU}{\mathbb{U}}
\newcommand{\tsV}{\mathbb{V}}
\newcommand{\tsW}{\mathbb{W}}
\newcommand{\tsX}{\mathbb{X}}
\newcommand{\tsY}{\mathbb{Y}}
\newcommand{\tsZ}{\mathbb{Z}}

%bf font for matrices
\newcommand{\bfA}{\mathbf{A}}
\newcommand{\bfB}{\mathbf{B}}
\newcommand{\bfC}{\mathbf{C}}
\newcommand{\bfD}{\mathbf{D}}
\newcommand{\bfE}{\mathbf{E}}
\newcommand{\bfF}{\mathbf{F}}
\newcommand{\bfG}{\mathbf{G}}
\newcommand{\bfH}{\mathbf{H}}
\newcommand{\bfI}{\mathbf{I}}
\newcommand{\bfJ}{\mathbf{J}}
\newcommand{\bfK}{\mathbf{K}}
\newcommand{\bfL}{\mathbf{L}}
\newcommand{\bfM}{\mathbf{M}}
\newcommand{\bfN}{\mathbf{N}}
\newcommand{\bfO}{\mathbf{O}}
\newcommand{\bfP}{\mathbf{P}}
\newcommand{\bfQ}{\mathbf{Q}}
\newcommand{\bfR}{\mathbf{R}}
\newcommand{\bfS}{\mathbf{S}}
\newcommand{\bfT}{\mathbf{T}}
\newcommand{\bfU}{\mathbf{U}}
\newcommand{\bfV}{\mathbf{V}}
\newcommand{\bfW}{\mathbf{W}}
\newcommand{\bfX}{\mathbf{X}}
\newcommand{\bfY}{\mathbf{Y}}
\newcommand{\bfZ}{\mathbf{Z}}

%bb font for standard spaces and expectation
\newcommand{\bbA}{\mathbb{A}}
\newcommand{\bbB}{\mathbb{B}}
\newcommand{\bbC}{\mathbb{C}}
\newcommand{\bbD}{\mathbb{D}}
\newcommand{\bbE}{\mathbb{E}}
\newcommand{\bbF}{\mathbb{F}}
\newcommand{\bbG}{\mathbb{G}}
\newcommand{\bbH}{\mathbb{H}}
\newcommand{\bbI}{\mathbb{I}}
\newcommand{\bbJ}{\mathbb{J}}
\newcommand{\bbK}{\mathbb{K}}
\newcommand{\bbL}{\mathbb{L}}
\newcommand{\bbM}{\mathbb{M}}
\newcommand{\bbN}{\mathbb{N}}
\newcommand{\bbO}{\mathbb{O}}
\newcommand{\bbP}{\mathbb{P}}
\newcommand{\bbQ}{\mathbb{Q}}
\newcommand{\bbR}{\mathbb{R}}
\newcommand{\bbS}{\mathbb{S}}
\newcommand{\bbT}{\mathbb{T}}
\newcommand{\bbU}{\mathbb{U}}
\newcommand{\bbV}{\mathbb{V}}
\newcommand{\bbW}{\mathbb{W}}
\newcommand{\bbX}{\mathbb{X}}
\newcommand{\bbY}{\mathbb{Y}}
\newcommand{\bbZ}{\mathbb{Z}}

%got font for any case
\newcommand{\gA}{\mathfrak{A}}
\newcommand{\gB}{\mathfrak{B}}
\newcommand{\gC}{\mathfrak{C}}
\newcommand{\gD}{\mathfrak{D}}
\newcommand{\gE}{\mathfrak{E}}
\newcommand{\gF}{\mathfrak{F}}
\newcommand{\gG}{\mathfrak{G}}
\newcommand{\gH}{\mathfrak{H}}
\newcommand{\gI}{\mathfrak{I}}
\newcommand{\gJ}{\mathfrak{J}}
\newcommand{\gK}{\mathfrak{K}}
\newcommand{\gL}{\mathfrak{L}}
\newcommand{\gM}{\mathfrak{M}}
\newcommand{\gN}{\mathfrak{N}}
\newcommand{\gO}{\mathfrak{O}}
\newcommand{\gP}{\mathfrak{P}}
\newcommand{\gQ}{\mathfrak{Q}}
\newcommand{\gR}{\mathfrak{R}}
\newcommand{\gS}{\mathfrak{S}}
\newcommand{\gT}{\mathfrak{T}}
\newcommand{\gU}{\mathfrak{U}}
\newcommand{\gV}{\mathfrak{V}}
\newcommand{\gW}{\mathfrak{W}}
\newcommand{\gX}{\mathfrak{X}}
\newcommand{\gY}{\mathfrak{Y}}
\newcommand{\gZ}{\mathfrak{Z}}

%old calligraphic font
\newcommand{\calA}{\mathcal{A}}
\newcommand{\calB}{\mathcal{B}}
\newcommand{\calC}{\mathcal{C}}
\newcommand{\calD}{\mathcal{D}}
\newcommand{\calE}{\mathcal{E}}
\newcommand{\calF}{\mathcal{F}}
\newcommand{\calG}{\mathcal{G}}
\newcommand{\calH}{\mathcal{H}}
\newcommand{\calI}{\mathcal{I}}
\newcommand{\calJ}{\mathcal{J}}
\newcommand{\calK}{\mathcal{K}}
\newcommand{\calL}{\mathcal{L}}
\newcommand{\calM}{\mathcal{M}}
\newcommand{\calN}{\mathcal{N}}
\newcommand{\calO}{\mathcal{O}}
\newcommand{\calP}{\mathcal{P}}
\newcommand{\calQ}{\mathcal{Q}}
\newcommand{\calR}{\mathcal{R}}
\newcommand{\calS}{\mathcal{S}}
\newcommand{\calT}{\mathcal{T}}
\newcommand{\calU}{\mathcal{U}}
\newcommand{\calV}{\mathcal{V}}
\newcommand{\calW}{\mathcal{W}}
\newcommand{\calX}{\mathcal{X}}
\newcommand{\calY}{\mathcal{Y}}
\newcommand{\calZ}{\mathcal{Z}}

%sf font for transposition and spaces like R
\newcommand{\sfA}{\mathsf{A}}
\newcommand{\sfB}{\mathsf{B}}
\newcommand{\sfC}{\mathsf{C}}
\newcommand{\sfD}{\mathsf{D}}
\newcommand{\sfE}{\mathsf{E}}
\newcommand{\sfF}{\mathsf{F}}
\newcommand{\sfG}{\mathsf{G}}
\newcommand{\sfH}{\mathsf{H}}
\newcommand{\sfI}{\mathsf{I}}
\newcommand{\sfJ}{\mathsf{J}}
\newcommand{\sfK}{\mathsf{K}}
\newcommand{\sfL}{\mathsf{L}}
\newcommand{\sfM}{\mathsf{M}}
\newcommand{\sfN}{\mathsf{N}}
\newcommand{\sfO}{\mathsf{O}}
\newcommand{\sfP}{\mathsf{P}}
\newcommand{\sfQ}{\mathsf{Q}}
\newcommand{\sfR}{\mathsf{R}}
\newcommand{\sfS}{\mathsf{S}}
\newcommand{\sfT}{\mathsf{T}}
\newcommand{\sfU}{\mathsf{U}}
\newcommand{\sfV}{\mathsf{V}}
\newcommand{\sfW}{\mathsf{W}}
\newcommand{\sfX}{\mathsf{X}}
\newcommand{\sfY}{\mathsf{Y}}
\newcommand{\sfZ}{\mathsf{Z}}

\title{Аппроксимация временными рядами конечного ранга}

\author{Звонарев Никита, гр. 522}
\institute[СПбГУ]{Санкт-Петербургский государственный университет \\
    Математико-механический факультет \\
    Кафедра статистического моделирования \\
    \vspace{0.2cm}
    Научный руководитель: к.ф.-м.н., доц. Голяндина Н. Э. \\
    \vspace{0.2cm}
    Рецензент: к.ф.-м.н., доц. Коробейников А. И. \\
    \vspace{0.2cm}
}
\date{
    Санкт-Петербург\\
    2015г.
}

\subject{Beamer}
% This is only inserted into the PDF information catalog. Can be left
% out.

% Delete this, if you do not want the table of contents to pop up at
% the beginning of each subsection:
% \AtBeginSubsection[]
% {
%   \begin{frame}<beamer>
%     \frametitle{Outline}
%     \tableofcontents[currentsection,currentsubsection]
%   \end{frame}
% }

\begin{document}

\begin{frame}
    \titlepage
\end{frame}

\begin{frame}
	\frametitle{Постановка задачи}
	\begin{itemize}
		\item Дано: $\tsX$ --- временной ряд, $\tsX \in \calX_N$ --- множество всех временных рядов длины $N$.
		\item $\calX_N^r$ --- подмножество $\calX_N$ рядов \structure{некоторой структуры}.
		\item Найти: $\tsY = (y_1, \ldots, y_N)$: $f_q(\tsY) \to \min, \tsY \in \calX_N^r$, $f_q(\tsY) = \sum \limits_{i=1}^N q_i(x_i - y_i)^2$, $q_i \ge 0$ --- неотрицательные веса.
		\item Приложение в статистике: наблюдаем $\tsX$ = $\tsY$ + $\tsN$, где $\tsY \in \calX_N^r$ --- сигнал, $\tsN$ --- белый гауссовский шум с нулевым матожиданием. Требуется найти оценку для $\tsY$. В качестве оценки можно рассматривать решение задачи.
	\end{itemize}
	
\end{frame}

\begin{frame}
	\frametitle{Ряды, управляемые ЛРФ}
	\begin{definition}
		Ряд, управляемый линейной рекуррентной формулой порядка~$r$: $\tsX = (x_1, \ldots, x_N), \quad x_n = \sum_{i = 1}^{r} a_i x_{n-i}, \quad n = r + 1, \ldots, N.$
	\end{definition}
	Параметрический вид:
	\begin{equation*}
	x_n = \sum_i P_i(n) \exp(\alpha_i n) \cos(2 \pi \omega_i n + \psi_i).
	\end{equation*}
	
	Можно показать эквивалентность поиска ряда, управляемого ЛРФ, и поиска ряда конечного $L$-ранга при наложении дополнительных условий.
\end{frame}

\begin{frame}
	\frametitle{Структура сигнала}
	
	Временной ряд $\tsX = (x_1, \ldots, x_N)$.
	
	Параметры: $1 < L < N$ --- \emph{длина окна}, $K = N - L + 1$, $0 \le r \le L$ --- ранг.
	
	Вектора вложения: $X_i = (x_i, \ldots, x_{i + L - 1})^\rmT, \qquad i = 1, \ldots, K$.
	\begin{definition}
		Траекторное пространство: $\spX^{(L)}(\tsX) = \spX^{(L)} = \mathspan(X_1, \ldots, X_K).$
	\end{definition}
	\begin{definition}
		$L$-ранг ряда $\tsX$ равен $r$ $\Leftrightarrow$ $\dim \spX^{(L)} = r$.
	\end{definition}
	
	$\calX_N$ --- множество всех временных рядов длины $N$. $\calX_N^r$ --- множество всех временных рядов длины $N$ $L$-ранга, не превосходящего $r$.
\end{frame}

\begin{frame}
	\frametitle{Постановка на матричном языке}
	\begin{enumerate}
		\item Временной ряд $\tsX = (x_1, \ldots, x_N)$.
		\item $\bfX = \calT(\tsX)$, где $\calT$ --- биективное отображение на множество траекторных матриц:
		\begin{equation*}
		\bfX = \begin{pmatrix}
		x_1 & x_2 & \cdots & x_K \\ 
		x_2 & \iddots & \iddots & x_{K+1} \\ 
		\vdots & \iddots & \vdots & \vdots \\ 
		x_L & x_{L+1} & \cdots & x_N
		\end{pmatrix} 
		\end{equation*}
		\item Найти $\bfY$: $||\bfX - \bfY||_? \to \min$, $\bfY \in \calH \cap \calM_r$, где $\calH$ --- множество ганкелевых матриц, $\calM_r$ --- матрицы ранга $\le r$.
		\item $\tsY = \calT^{-1}(\bfY)$
	\end{enumerate}
\end{frame}

\section{Способ решения}
\subsection{Теоретические сведения}
\begin{frame}
	\frametitle{Стандартное решение}
	$\bfY$: $||\bfX - \bfY||_? \to \min$, $\bfY \in \calH \cap \calM_r$, где $\calH \subset \sfR^{L \times K}$ --- множество ганкелевых матриц, $\calM_r \subset \sfR^{L \times K}$ --- матрицы ранга~$\le r$.
	
	\begin{solution}
		
		Переменные проекции: 
		\begin{equation*}
		\bfY = (\Pi_\calH \circ \Pi_{\calM_r})^I (\bfX),
		\end{equation*}
		$\Pi_\calH$ --- проектор на ганкелевы матрицы, $\Pi_{\calM_r}$ --- проектор на матрицы ранга $\le r$.
	\end{solution}
	\begin{enumerate}
	\item Интересует сходимость к нужному множеству
	\item Выбор норм тоже интересен
    \end{enumerate}
\end{frame}

\begin{frame}
	\frametitle{Предварительные утверждения}
	$\bfY = (\Pi_\calH \circ \Pi_{\calM_r})^I (\bfX)$, $\Pi_\calH$ --- проектор на ганкелевы матрицы, $\Pi_{\calM_r}$ --- проектор на матрицы ранга $\le r$ по норме $||\cdot||$.
	
    \vspace{0.3cm}
	$\calH$ --- линейное подпространство. $\calM_r$ --- ни линейное, ни выпуклое множество. Тем не менее, $\calM_r$ замкнуто относительно умножения на константу, т.е. если $\bfZ \in \calM_r$, то $a \bfZ \in \calM_r$ для любого $a \in \sfR$.
	\begin{proposition}
		Пусть $\calX$ --- гильбертово пространство, $\calM \subset \calX$ --- замкнутое относительно умножения на константу подпространство, $\Pi_\calM$ --- проектор на $\calM$. Тогда для любого $x \in \calX$ выполнена теорема Пифагора: $\|x\|^2~=~\|x~-~\Pi_\calM x\|^2~+~\|\Pi_\calM x\|^2$.
	\end{proposition}
\end{frame}

\begin{frame}
	\frametitle{Доказательство утверждения}
	\begin{proposition}
		Пусть $\calX$ --- гильбертово пространство, $\calM \subset \calX$ --- замкнутое относительно умножения на константу подпространство, $\Pi_\calM$ --- проектор на $\calM$. Тогда для любого $x \in \calX$ выполнена теорема Пифагора: $\|x\|^2~=~\|x~-~\Pi_\calM x\|^2~+~\|\Pi_\calM x\|^2$.
	\end{proposition}
	
	\begin{proof}
	По условию, множество $\calM$ можно представить как $\calM = \bigcup\limits_{l \in \calL}l$, где $\calL$ это множество всех прямых, лежащих в $\calM$ и проходящих через $0$. Тогда проекцию можно описать так: вначале мы выбираем прямую $l$ такую, что $\dist(x, l) \rightarrow \min\limits_{l \in \calL}$; далее, $y = \Pi_\calM x$ --- ортогональная проекция $x$ на прямую $l$, которая является линейным подпространством.
	\end{proof}	
\end{frame}

\begin{frame}
	\frametitle{Замкнутость $\calM_r$}
	\begin{proposition} \small
		Множество матриц ранга $\le r$ $\calM_r$ замкнуто в топологии, порожденной фробениусовской нормой.
	\end{proposition}
	
	\begin{proof} \small
		Рассмотрим функцию-``индикатор'' $f: \sfR^{L \times K} \to \sfR$, определенную как \begin{equation*}
		f(\bfX) = \sum_{\bfU \in \sfU(\bfX)} |\det(\bfU)|,
		\end{equation*}
		где $\sfU(\bfX)$ --- множество всевозможных квадратных подматриц порядка $r+1$ матрицы $\bfX$. Заметим, что $f$ --- непрерывная, и $\bfX \in \calM_r \Leftrightarrow f(\bfX) = 0$.
		
		$\bfX_1, \bfX_2, \ldots$ --- сходящаяся последовательность, $\bfX_k \in \calM_r$ $\forall k$, $\bfX'$ --- ее предел. Тогда $f(\bfX_k) = 0$ $\forall k$; используя непрерывность $f$, получаем $f(\bfX') = 0$.
	\end{proof}	
\end{frame}

\begin{frame}
	\frametitle{Основная теорема}
	Рассмотрим один шаг алгоритма:
	$\bfY_{k+1}=\Pi_\calH \Pi_{\calM_r} \bfY_{k}, \mbox{\ где\ } \bfY_{0}=\bfX.$
	\begin{theorem}
			 Пусть множество $\calM_r$ замкнуто в топологии, порожденной нормой $\|\cdot\|$. Тогда
			\begin{enumerate}
				\item $\|\bfY_k - \Pi_{\calM_r}\bfY_k\| \to 0$ при $k \to +\infty$, $\|\Pi_{\calM_r}\bfY_k - \bfY_{k+1}\| \to 0$ при $k \to +\infty$.
				\item Существует сходящаяся подпоследовательность матриц $\bfY_{i_1}, \bfY_{i_2}, \ldots$ такая, что ее предел $\bfY^*$  лежит в $\calM_r \cap \calH$.
			\end{enumerate}
	\end{theorem}
\end{frame}

\begin{frame}
	\frametitle{Доказательство теоремы (первый пункт)}
	\begin{proof} \small
    Используем следующее неравенство:
    \begin{equation*}
    \|\bfY_k - \Pi_{\calM_r} \bfY_k\| \ge \|\Pi_{\calM_r} \bfY_k - \bfY_{k + 1}\| \ge \|\bfY_{k+1} - \Pi_{\calM_r} \bfY_{k + 1}\|.
    \end{equation*}
    Согласно неравенству, последовательности $\|\bfY_k~-~\Pi_{\calM_r} \bfY_k\|$ и $\|\Pi_{\calM_r} \bfY_k - \bfY_{k + 1}\|$, $k = 1, 2, \ldots$, невозрастающие. Очевидно, что снизу они ограничены нулем.  Следовательно, они имеют один и тот же предел $c$.
	
    Докажем, что $c = 0$, предполагая противное $c > 0$. Тогда существует $d > 0$ такое, что $\|\bfY_k - \Pi_{\calM_r} \bfY_k\| > d$ и $\|\Pi_{\calM_r} \bfY_k - \bfY_{k + 1}\| > d$ для любых $k = 1, 2, \ldots$. Согласно первому предложению, следующее выражение верно:
	$\|\bfY_k\|^2 = \|\bfY_k - \Pi_{\calM_r} \bfY_k\|^2 + \|\Pi_{\calM_r} \bfY_k - \bfY_{k + 1}\|^2 + \|\bfY_{k + 1}\|^2.$
	Таким образом, $\|\bfY_{k+1}\|^2 < \|\bfY_k\|^2 - 2d^2$. По индукции получаем $\|\bfY_{k+j}\|^2 < \|\bfY_k\|^2 - 2 j d^2$ для любых $j = 1, 2, \ldots$. Выберем $k = 1$, и $j = \lceil \|\bfY_k\|^2 / (2d^2) \rceil + 1$. Тогда $\|\bfY_{k+j}\|^2 < 0$, что невозможно.
	\end{proof}
\end{frame}

\begin{frame}
	\frametitle{Доказательство теоремы (второй пункт)}
	\begin{proof}
		Рассмотрим последовательность $(\Pi_{\calM_r} \bfY_k,\, k = 1, 2, \ldots)$. Она ограничена, потому что $\|\Pi_{\calM_r} \bfZ\| \le \|\bfZ\|$ и $\|\Pi_{\calH} \bfZ\| \le \|\bfZ\|$ для любых $\bfZ \in \sfR^{L \times K}$ (согласно первому предложению). Тогда из нее можно выбрать сходящуюся подпоследовательность $(\Pi_{\calM_r} \bfY_{i_k})$; обозначим ее предел как $\bfY^*\in\calM_r$, где $\|\Pi_{\calM_r} \bfY_{i_k} - \bfY_{i_k + 1}\| = \|\Pi_{\calM_r} \bfY_{i_k} - \Pi_\calH \Pi_{\calM_r} \bfY_{i_k}\| \to 0$ при $k \to + \infty$. Используя тот факт, что $\|\bfZ - \Pi_\calH \bfZ\|$ это композиция непрерывных отображений, мы получаем, что $\|\bfY^* - \Pi_\calH \bfY^*\| = 0$, $\bfY^* \in \calM_r \cap \calH$. Наконец, $\Pi_\calH$ это непрерывное отображение, и поэтому последовательность $(\Pi_\calH \Pi_{\calM_r} \bfY_{i_k})$ сходится к $\bfY^*$. Таким образом, $\bfY_{i_k + 1}$ --- требуемая подпоследовательность.
	\end{proof}
\end{frame}

\begin{frame}
	\frametitle{Замечания}
	Замкнутость $\calM_r$ была доказана для фробениусовской нормы. Естественно, она сохраняется и для любой эквивалентной нормы.
	
	В дальнейшем, мы будем рассматривать нормы, порожденные следующими взвешенными фробениусовскими скалярными произведениями: 
	\begin{equation*}
	\langle\bfY, \bfZ\rangle_M = \sum_{l = 1}^L \sum_{k = 1}^K m_{l, k} y_{l, k} z_{l, k}, \quad \text{где все} \; m_{l,k} \ge 0.
	\end{equation*}
	Для эквивалентности норм необходимо, чтобы все $m_{l,k} > 0$.
	
\end{frame}

\subsection{Нормы и алгоритмы}
\begin{frame}
	\frametitle{Стандартное решение}
	$\bfY$: $||\bfX - \bfY||_? \to \min$, $\bfY \in \calH \cap \calM_r$, где $\calH$ --- множество ганкелевых матриц, $\calM_r$ --- матрицы ранга $\le r$.
	
	\begin{solution}
		
		Переменные проекции: 
		\begin{equation*}
		\bfY = (\Pi_\calH \circ \Pi_{\calM_r})^I (\bfX),
		\end{equation*}
		$\Pi_\calH$ --- проектор на ганкелевы матрицы, $\Pi_{\calM_r}$ --- проектор на матрицы ранга $\le r$.
	\end{solution}
	
	$\Pi_\calH$ и $\Pi_{\calM_r}$ зависят от $||\cdot||_?$. Варианты:
	\begin{enumerate}
		\item $||\bfX||_M^2 = \sum_{l = 1}^L \sum_{k = 1}^K m_{l, k} (x_{l, k})^2$, все $m_{l, k} \ge 0$
		\item Частный случай предыдущего пункта: $m_{l, k} = 1$.
		\item $||\bfX||_\bfC^2 = \text{tr}(\bfX \bfC \bfX^\rmT)$, $\bfC$ --- симметричная, неотрицательно определенная матрица порядка $K \times K$.
	\end{enumerate}
\end{frame}

\begin{frame}
	\frametitle{Эквивалентность скалярных произведений и норм}
	\begin{enumerate}
		\item $<\tsX, \tsY>_q = \sum_{i=1}^N q_i x_i  y_i$
		\item $<\bfX, \bfY>_M = \sum_{l = 1}^L \sum_{k = 1}^K m_{l, k} x_{l, k} y_{l, k}$
		\item $<\bfX, \bfY>_\bfC = \text{tr}(\bfX \bfC \bfY^\rmT)$.
	\end{enumerate}
	\begin{proposition}
		\small
	    \begin{enumerate}
		\item Пусть $\bfX = \calT(\tsX)$,  $\bfY = \calT(\tsY)$. Тогда $<\tsX,\tsY>_q= <\bfX,\bfY>_M$ тогда и только тогда, когда
		\begin{equation*}
		q_i = \sum_{\substack{1 \le l \le L \\ 1 \le k \le K \\ l+k-1=i}} m_{l,k}.
		\end{equation*}
		
		\item Для диагональной матрицы $\bfC$, $<\bfX,\bfY>_M= <\bfX,\bfY>_\bfC$ тогда и только тогда, когда
		\begin{equation*}
		m_{l,k}=c_{k,k}.
		\end{equation*}
	    \end{enumerate}
	\end{proposition}
	
\end{frame}

\begin{frame}
	\frametitle{Проектор $\Pi_{\calM_r}$}
	Варианты:
	\begin{enumerate}
		\item $||\bfX||_M$, все $m_{l, k} = 1$: через стандартное SVD-разложение. $\Pi_{\calM_r}(\bfX) = \bfU \Sigma_r \bfV^\rmT$, где $\bfX = \bfU \Sigma \bfV^\rmT$.
		\vspace{0.2cm}
		\item $||\bfX||_M$, общий случай: EM-подобный алгоритм.
		\vspace{0.2cm}
		\item $||\bfX||_\bfC$: косоугольное SVD-разложение. $\bfC = \bfO_{\bfC}^\rmT \bfO_\bfC$, $\bfB = \bfX \bfO_{\bfC}^\rmT$, $\Pi_{\calM_r}^\bfC(\bfX) = \Pi_{\calM_r}(\bfB)(\bfO_{\bfC}^\rmT)^\dagger$.
	\end{enumerate}
\end{frame}

\begin{frame}
	\frametitle{Проектор $\Pi_{\calH}$}
	Варианты:
	\begin{enumerate}
		\item $||\bfX||_M$, $m_{l,k}$ на побочных диагоналях равны, в частности $m_{l, k} = 1$: диагональное усреднение.
		\begin{equation*}
		\bfX = \Pi_{\calH}(\bfY), \quad x_{l,k} = \sum_{i + j = l + k} y_{i, j} / w_{l + k - 1}.
		\end{equation*}
		\item $||\bfX||_\bfC$, $\bfC$ --- диагональная: взвешенное диагональное усреднение с весами $c_{i, i}$, где $\bfC = (c_{i, j})$.
		
		\begin{equation*}
		\bfX = \Pi_{\calH}(\bfY), \quad x_{l,k} = \frac{\sum_{i,j:\, i+j=l+k} c_{i,i} y_{i,j}}{\sum_{i,j:\, i+j=l+k} c_{i,i}}.
		\end{equation*}
	\end{enumerate}
\end{frame}

\begin{frame}
	\frametitle{Варианты алгоритма Cadzow}
	$||\bfX||^2_M =  \sum_{l = 1}^L \sum_{k = 1}^K m_{l, k}$
	\begin{enumerate}
		\item Все $m_{l, k} = 1$ --- базовый алгоритм Cadzow ($\Pi_{\calM_r}$ --- обычный SVD, $\Pi_\calH$ --- диаг. усреднение).
		\begin{thnote}
			Эквивалентные веса базового алгоритма:
			\begin{equation*}
			q_i = w_i = \begin{cases}
			i & \text{для $i = 1, \ldots, L-1,$}\\
			L & \text{для $i = L, \ldots, K,$}\\
			N - i + 1 & \text{для $i = K + 1, \ldots, N,$}
			\end{cases} \qquad.
			\end{equation*}
			что не всегда естественно.
		\end{thnote}
		\item $||\bfX||_M$, $m_{l, k} = 1 / w_{l + k - 1}$ --- Weighted Cadzow ($\Pi_{\calM_r}$ --- EM, $\Pi_\calH$ --- диаг. усреднение). Для него все $q_i$ = 1.
	\end{enumerate}
\end{frame}

\begin{frame}
	\frametitle{Варианты Cadzow с косоугольным SVD}
	$||\bfX||_\bfC^2 = \text{tr}(\bfX \bfC \bfX^\rmT)$
	\begin{enumerate}
		\item $\bfC = \text{diag}(1, \alpha, \alpha, \ldots, \alpha, 1, \alpha, \ldots, 1)$,
		где единицы стоят на $1$, $L + 1$, $2L + 1$, \ldots , $K$ месте, $0 \le \alpha \le 1$ (Cadzow($\alpha$))
		\begin{theorem}
			Эквивалентные веса алгоритма Cadzow($\alpha$):
			\begin{equation*}
			q_i = \begin{cases}
			1 + (i - 1) \alpha & \text{для $i = 1, \ldots, L-1,$}\\
			1 + (L - 1) \alpha & \text{для $i = L, \ldots, K-1,$}\\
			1 + (N - i) \alpha & \text{для $i = K, \ldots, N.$}
			\end{cases}
			\end{equation*}
		\end{theorem}
		\item $\hat \bfC = \text{diag}(C)$, где $C$ --- усреднение матрицы $\bfM$, примененной в алгоритме Weighted Cadzow, по столбцам (Hat-Cadzow).
	\end{enumerate}
	
	$\Pi_{\calM_r}$ --- косоугольное SVD, $\Pi_\calH$ --- взвешенное диаг. усреднение
\end{frame}

\begin{frame}
	\frametitle{Пример весов ряда ($N = 40$, $L = 8$)}
	\begin{center}
		\includegraphics*[width = 10cm]{weights.pdf}
	\end{center}	
\end{frame}

\begin{frame}
	\frametitle{Алгоритм Extended Cadzow}
	\begin{equation*}
	\calX = \begin{pmatrix}
	&  &  & x_1 & x_2 & \cdots & x_K & x_{K+1} & \cdots & x_N \\ 
	&  & \iddots & x_2 & x_3 & \iddots & x_{K+1} & \iddots & \iddots &  \\ 
	& x_1 & \iddots & \iddots & \iddots & \iddots & \iddots & x_N &  &  \\ 
	x_1 & x_2 & \cdots & x_L & x_{L+1} & \cdots & x_N &  &  & 
	\end{pmatrix}.
	\end{equation*}
	\begin{enumerate}
		\item $q_i = 1$, $i = 1, \ldots, N$, но задача несколько иная
		\item $\tilde \calT$ : $\tilde \calT(\tsX) = \calX$ --- биекция между рядами и матрицами с пропусками.
		\item Проектор на множество матриц неполного ранга $\Pi_{\widetilde \calM_r}$ (EM-алгоритм).
		\item Проектор на множество траекторных псевдоматриц $\Pi_\gX$ (диагональное усреднение).
		\item Сам алгоритм: $\tsY = \tilde \calT^{-1}((\Pi_\gX \circ \Pi_{\widetilde \calM_r})^I (\tilde \calT(\tsX)))$.
	\end{enumerate}
\end{frame}

\begin{frame}
	\frametitle{Различные проблемы}
	\begin{itemize}
		\item Базовый алгоритм Cadzow --- имеет $q_i$, далекие от оптимальных
		\item Weighted, Extended Cadzow --- работает долго, так как использует итерационный EM-алгоритм для проектирования на множество матриц ранга $\le r$
		\item Cadzow($\alpha$) с косоугольным SVD: при $\alpha = 0$~--- вообще нет сходимости к нужному множеству, $\alpha \approx 0$, $\alpha > 0$: медленно сходится, проблемы со слабой разделимостью, требуется, чтобы $N$ было кратно $L$
	\end{itemize}
	\begin{equation*}
	\alpha = 0 \implies \bfC = \text{diag}(1, 0, 0, \ldots, 0, 1, 0, \ldots, 1) \quad \text{--- }
	\end{equation*}
	вырожденная матрица
\end{frame}

\section{Слабая разделимость}
\begin{frame} \small
	\frametitle{Слабая разделимость в косоугольном случае}
 $\bfC \in \sfR^{K \times K}$ --- симметричная неотрицательно определенная, $\tsX_1$ и $\tsX_2$~--- два временных ряда длины $N$, $\bfX^1$, $\bfX^2$ --- их траекторные матрицы. Определим \emph{коэффициент корреляции $i$-го и $j$-го столбца}:
\begin{equation*}
\rho^c_{i,j} = \frac{(X^1_i, X^2_j)}{\|X^1_i\| \|X^2_j\|},
\end{equation*}
где $X^k_i$ --- $i$-й столбец матрицы $\bfX^k$, $k = 1, 2$. 

\emph{Коэффициент корреляции $i$-й и $j$-й строчки}:
\begin{equation*}
\rho^r_{i,j} = \frac{(X^{1,i}, X^{2,j})_\bfC}{\|X^{1,i}\|_\bfC \|X^{2,j}\|_\bfC},
\end{equation*}
где $X^{k,i}$ --- $i$-я строка матрицы $\bfX^k$, $k = 1, 2$, а $(\cdot, \cdot)_\bfC$ --- косоугольное скалярное произведение в $\sfR^K$, порожденное матрицей $\bfC$: $(X, Y)_\bfC = X \bfC Y^\sfT$ (здесь $X$, $Y$ --- вектор-строчки), $\| \cdot \|_\bfC$ --- соответствующая норма.
\end{frame}

\begin{frame}
	\frametitle{Слабая разделимость гармоники и константы}
	\begin{definition}
		Скажем, что ряды $\tsX_1$ и $\tsX_2$ \emph{слабо $\varepsilon$-разделимы} если
		\begin{equation*}
			\rho = \max\Big(\max_{1 \le i,j \le K}|\rho^c_{i,j}|, \max_{1 \le i,j \le L}|\rho^r_{i,j}|\Big) < \varepsilon.
		\end{equation*}
	\end{definition}
	Интересует порядок $\varepsilon$ при $N \to \infty$, если брать первые $N$ точек бесконечного ряда.
	
	Будем рассматривать $\tsX_1^\infty = (\cos(2 \pi \omega k), k = 1, 2, \ldots)$ и $\tsX_2^\infty = (c, c, \ldots)$ при $L, K \to \infty$, где $N = L + K - 1$.
	
	\vspace{0.3cm}
	Известный результат: когда $\bfC$ --- единичная матрица, то $\varepsilon$ имеет порядок $1/\min(L,K)$.
\end{frame}

\begin{frame}
\begin{proposition}
	\frametitle{Слабая разделимость: новые результаты}
	Пусть $\tsX_1^\infty = (\cos(2 \pi \omega k), k = 1, 2, \ldots)$, где $0<\omega <0.5$ --- синусоида, $\tsX_2^\infty = (c, c, \ldots)$ --- константа,  $L,K \to \infty$ так, что $h = h_L = N/L$, где $N=L+K-1$, целое, и $\bfC=\bfC(\alpha)$ диагональная с диагональными элементами (алгоритм Cadzow($\alpha$)):
	\begin{equation*}
	c_k = \begin{cases}
	1, & \text{если} \quad k = jL+1 \quad \text{для некоторых} \ j = 0, \ldots, h-1,\\
	\alpha, & \text{в противном случае},
	\end{cases}
	\end{equation*}
	где $0 \le \alpha \le 1$. Тогда $\rho$ имеет порядок $\max\left(\frac{1}{L}, \frac{(1-\alpha)C_{L,K}+\alpha}{(1-\alpha)N/L+\alpha K}\right)$, где порядок $C_{L,K}$
	может меняться от $O(1)$ to $O(N/L)$ в зависимости от того, как $K$ стремится к бесконечности.
	\end{proposition}
    \small При $\alpha \approx 0$, $C_{L, K} = O(1)$: оптимальное $L \approx \sqrt{N}$, а $\rho$ имеет порядок $ 1/\sqrt{N}$.
\end{frame}
 
\section{Эксперименты}
\begin{frame}
	\frametitle{Пример №1 (одна итерация)}
	Задача оценки сигнала: $N = 40$, $L = 20$, $r = 2$, $\tsX = (x_{1}, \ldots, x_N),$  $x_k = 5\sin{\frac{2 k \pi}{6}}$.
	\begin{center}
		\includegraphics*[width = 9cm]{s1_it1.pdf}
	\end{center}
	
	
\end{frame}

\begin{frame}
	\frametitle{Пример №2 (много итераций)}
	Задача оценки сигнала: $N = 40$, $L = 20$, $r = 2$, $\tsX = (x_{1}, \ldots, x_N),$  $x_k = 5\sin{\frac{2 k \pi}{6}}$.
	
	\begin{center}
		\includegraphics*[width = 9cm]{s1_it100.pdf}
	\end{center}
	
\end{frame}

\begin{frame}
	\frametitle{Пример №3}
	Сходимость алгоритма: $N = 40$, $L_1 = 8$, $L_2 = 20$, $r = 2$, $\tsX = (x_{1}, \ldots, x_N),$  $x_k = 5\sin{\frac{2 k \pi}{6}}$.
	
	\begin{center}
		\includegraphics*[width = 9cm]{cadzowspeed_2.pdf}
	\end{center}
	
\end{frame}
\section{Итоги}

%\begin{frame}
%  \frametitle{Выводы}
%  Рассмотрен широкий класс алгоритмов, проведено сравнение на модельном примере:
%  \begin{itemize}
%  \item Доказана сходимость по подпоследовательностям для всех алгоритмов
%  \item Единичные веса достигаются только на алгоритмах с вложенными итерациями (Extended, Weighted Cadzow), которые являются медленными.
%  \item Алгоритмы с более близкими к единичным весами дают лучшую аппроксимацию в пределе, но медленее сходятся и имеют проблемы с разделимостью:
%  
%  Cadzow($\alpha$): малое $L$, $\alpha$ $\Rightarrow$ плохая разделимость
%  \item Лучший метод среди всех -- Extended Cadzow. Среди методов без внутренних итераций на одной лучше всего работает Cadzow($1$) и Hat-Cadzow, в пределе --- Cadzow($0.1$)
%  \end{itemize}

%\end{frame}

\begin{frame}
	\frametitle{Выводы}\footnotesize
	\begin{itemize}
		\item Рассмотрен широкий класс итерац. алгоритмов решения задачи 
		%(повторить для рядов с весами $q_i$) 
		для оценивания сигнала $\tsY$  в модели $\tsX = \tsY + \tsN$.
		\item Доказана сходимость по подпоследовательностям для всех алгоритмов
		\item Единичные веса $q_i$ достигаются только на алгоритмах с вложенными итерациями (Extended, Weighted Cadzow), которые являются медленными.
		\item  На численных примерах проведено исследование и получены следующие результаты:
		\begin{itemize} \footnotesize
			\item Лучший метод среди всех -- Extended Cadzow.
			\item Алгоритмы с более близкими к единичным весами дают лучшую аппроксимацию в пределе, но медленнее сходятся и имеют проблемы с разделимостью,
			
			например, Cadzow($\alpha$) при маленькой  длине окна $L$ или малом параметре $\alpha$.
			\item На одной итерации лучше всего работает Cadzow($1$) (= Singular Spectrum Analysis) и Hat-Cadzow, т.е. методы с лучшей разделимостью.
		\end{itemize}
	\end{itemize}
	
\end{frame}

\end{document}