\documentclass[12pt,a4paper]{article}

\usepackage[text={14cm,20cm}]{geometry}
\usepackage{amsmath,amssymb,amsthm,amscd,amsfonts}
\usepackage[utf8x]{inputenc}

\usepackage[T2A]{fontenc}
\usepackage[russian]{babel}


\usepackage{euscript}
\usepackage{relsize}
\usepackage{mathdots}
\usepackage{graphicx}
%\usepackage{epstopdf}
\usepackage{indentfirst}
\usepackage{empheq}
\usepackage{multirow}

\newcommand*\widefbox[1]{\fbox{\hspace{2em}#1\hspace{2em}}}

\usepackage[colorlinks, urlcolor=blue, pdfborder={0 0 0 [0 0]}]{hyperref}

\hyphenation{Struc-tu-red}
\hyphenation{Ran-do-mized}
\hyphenation{Ma-xi-mi-za-tion}
\DeclareMathOperator*{\argmax}{arg\,max}
\DeclareMathOperator*{\argmin}{arg\,min}
\DeclareMathOperator{\tr}{tr}
\providecommand*{\BibDash}{}

\def\rank{\mathop{\mathrm{rank}}}
\def\rev{\mathop{\mathrm{rev}}}
\DeclareSymbolFont{bbold}{U}{bbold}{m}{n}
\DeclareSymbolFontAlphabet{\mathbbold}{bbold}

\newtheorem{corollary}{Следствие}
\newtheorem{proposition}{Предложение}
\newtheorem{algorithm}{Алгоритм}
\newtheorem{theorem}{Теорема}
\newtheorem{lemma}{Лемма}
\newtheorem{remark}{Замечание}
\newtheorem{problem}{Задача}

%\usepackage{euscript}

%new calligraphic font for subspaces
\usepackage{euscript}
\newcommand{\spA}{\EuScript{A}}
\newcommand{\spB}{\EuScript{B}}
\newcommand{\spC}{\EuScript{C}}
\newcommand{\spD}{\EuScript{D}}
\newcommand{\spE}{\EuScript{E}}
\newcommand{\spF}{\EuScript{F}}
\newcommand{\spG}{\EuScript{G}}
\newcommand{\spH}{\EuScript{H}}
\newcommand{\spI}{\EuScript{I}}
\newcommand{\spJ}{\EuScript{J}}
\newcommand{\spK}{\EuScript{K}}
\newcommand{\spL}{\EuScript{L}}
\newcommand{\spM}{\EuScript{M}}
\newcommand{\spN}{\EuScript{N}}
\newcommand{\spO}{\EuScript{O}}
\newcommand{\spP}{\EuScript{P}}
\newcommand{\spQ}{\EuScript{Q}}
\newcommand{\spR}{\EuScript{R}}
\newcommand{\spS}{\EuScript{S}}
\newcommand{\spT}{\EuScript{T}}
\newcommand{\spU}{\EuScript{U}}
\newcommand{\spV}{\EuScript{V}}
\newcommand{\spW}{\EuScript{W}}
\newcommand{\spX}{\EuScript{X}}
\newcommand{\spY}{\EuScript{Y}}
\newcommand{\spZ}{\EuScript{Z}}

%font for text indices like transposition X^\mathrm{T}
\newcommand{\rmA}{\mathrm{A}}
\newcommand{\rmB}{\mathrm{B}}
\newcommand{\rmC}{\mathrm{C}}
\newcommand{\rmD}{\mathrm{D}}
\newcommand{\rmE}{\mathrm{E}}
\newcommand{\rmF}{\mathrm{F}}
\newcommand{\rmG}{\mathrm{G}}
\newcommand{\rmH}{\mathrm{H}}
\newcommand{\rmI}{\mathrm{I}}
\newcommand{\rmJ}{\mathrm{J}}
\newcommand{\rmK}{\mathrm{K}}
\newcommand{\rmL}{\mathrm{L}}
\newcommand{\rmM}{\mathrm{M}}
\newcommand{\rmN}{\mathrm{N}}
\newcommand{\rmO}{\mathrm{O}}
\newcommand{\rmP}{\mathrm{P}}
\newcommand{\rmQ}{\mathrm{Q}}
\newcommand{\rmR}{\mathrm{R}}
\newcommand{\rmS}{\mathrm{S}}
\newcommand{\rmT}{\mathrm{T}}
\newcommand{\rmU}{\mathrm{U}}
\newcommand{\rmV}{\mathrm{V}}
\newcommand{\rmW}{\mathrm{W}}
\newcommand{\rmX}{\mathrm{X}}
\newcommand{\rmY}{\mathrm{Y}}
\newcommand{\rmZ}{\mathrm{Z}}

%tt font for time series
\newcommand{\tsA}{\mathbb{A}}
\newcommand{\tsB}{\mathbb{B}}
\newcommand{\tsC}{\mathbb{C}}
\newcommand{\tsD}{\mathbb{D}}
\newcommand{\tsE}{\mathbb{E}}
\newcommand{\tsF}{\mathbb{F}}
\newcommand{\tsG}{\mathbb{G}}
\newcommand{\tsH}{\mathbb{H}}
\newcommand{\tsI}{\mathbb{I}}
\newcommand{\tsJ}{\mathbb{J}}
\newcommand{\tsK}{\mathbb{K}}
\newcommand{\tsL}{\mathbb{L}}
\newcommand{\tsM}{\mathbb{M}}
\newcommand{\tsN}{\mathbb{N}}
\newcommand{\tsO}{\mathbb{O}}
\newcommand{\tsP}{\mathbb{P}}
\newcommand{\tsQ}{\mathbb{Q}}
\newcommand{\tsR}{\mathbb{R}}
\newcommand{\tsS}{\mathbb{S}}
\newcommand{\tsT}{\mathbb{T}}
\newcommand{\tsU}{\mathbb{U}}
\newcommand{\tsV}{\mathbb{V}}
\newcommand{\tsW}{\mathbb{W}}
\newcommand{\tsX}{\mathbb{X}}
\newcommand{\tsY}{\mathbb{Y}}
\newcommand{\tsZ}{\mathbb{Z}}

%bf font for matrices
\newcommand{\bfA}{\mathbf{A}}
\newcommand{\bfB}{\mathbf{B}}
\newcommand{\bfC}{\mathbf{C}}
\newcommand{\bfD}{\mathbf{D}}
\newcommand{\bfE}{\mathbf{E}}
\newcommand{\bfF}{\mathbf{F}}
\newcommand{\bfG}{\mathbf{G}}
\newcommand{\bfH}{\mathbf{H}}
\newcommand{\bfI}{\mathbf{I}}
\newcommand{\bfJ}{\mathbf{J}}
\newcommand{\bfK}{\mathbf{K}}
\newcommand{\bfL}{\mathbf{L}}
\newcommand{\bfM}{\mathbf{M}}
\newcommand{\bfN}{\mathbf{N}}
\newcommand{\bfO}{\mathbf{O}}
\newcommand{\bfP}{\mathbf{P}}
\newcommand{\bfQ}{\mathbf{Q}}
\newcommand{\bfR}{\mathbf{R}}
\newcommand{\bfS}{\mathbf{S}}
\newcommand{\bfT}{\mathbf{T}}
\newcommand{\bfU}{\mathbf{U}}
\newcommand{\bfV}{\mathbf{V}}
\newcommand{\bfW}{\mathbf{W}}
\newcommand{\bfX}{\mathbf{X}}
\newcommand{\bfY}{\mathbf{Y}}
\newcommand{\bfZ}{\mathbf{Z}}

%bb font for standard spaces and expectation
\newcommand{\bbA}{\mathbb{A}}
\newcommand{\bbB}{\mathbb{B}}
\newcommand{\bbC}{\mathbb{C}}
\newcommand{\bbD}{\mathbb{D}}
\newcommand{\bbE}{\mathbb{E}}
\newcommand{\bbF}{\mathbb{F}}
\newcommand{\bbG}{\mathbb{G}}
\newcommand{\bbH}{\mathbb{H}}
\newcommand{\bbI}{\mathbb{I}}
\newcommand{\bbJ}{\mathbb{J}}
\newcommand{\bbK}{\mathbb{K}}
\newcommand{\bbL}{\mathbb{L}}
\newcommand{\bbM}{\mathbb{M}}
\newcommand{\bbN}{\mathbb{N}}
\newcommand{\bbO}{\mathbb{O}}
\newcommand{\bbP}{\mathbb{P}}
\newcommand{\bbQ}{\mathbb{Q}}
\newcommand{\bbR}{\mathbb{R}}
\newcommand{\bbS}{\mathbb{S}}
\newcommand{\bbT}{\mathbb{T}}
\newcommand{\bbU}{\mathbb{U}}
\newcommand{\bbV}{\mathbb{V}}
\newcommand{\bbW}{\mathbb{W}}
\newcommand{\bbX}{\mathbb{X}}
\newcommand{\bbY}{\mathbb{Y}}
\newcommand{\bbZ}{\mathbb{Z}}

%got font for any case
\newcommand{\gA}{\mathfrak{A}}
\newcommand{\gB}{\mathfrak{B}}
\newcommand{\gC}{\mathfrak{C}}
\newcommand{\gD}{\mathfrak{D}}
\newcommand{\gE}{\mathfrak{E}}
\newcommand{\gF}{\mathfrak{F}}
\newcommand{\gG}{\mathfrak{G}}
\newcommand{\gH}{\mathfrak{H}}
\newcommand{\gI}{\mathfrak{I}}
\newcommand{\gJ}{\mathfrak{J}}
\newcommand{\gK}{\mathfrak{K}}
\newcommand{\gL}{\mathfrak{L}}
\newcommand{\gM}{\mathfrak{M}}
\newcommand{\gN}{\mathfrak{N}}
\newcommand{\gO}{\mathfrak{O}}
\newcommand{\gP}{\mathfrak{P}}
\newcommand{\gQ}{\mathfrak{Q}}
\newcommand{\gR}{\mathfrak{R}}
\newcommand{\gS}{\mathfrak{S}}
\newcommand{\gT}{\mathfrak{T}}
\newcommand{\gU}{\mathfrak{U}}
\newcommand{\gV}{\mathfrak{V}}
\newcommand{\gW}{\mathfrak{W}}
\newcommand{\gX}{\mathfrak{X}}
\newcommand{\gY}{\mathfrak{Y}}
\newcommand{\gZ}{\mathfrak{Z}}

%old calligraphic font
\newcommand{\calA}{\mathcal{A}}
\newcommand{\calB}{\mathcal{B}}
\newcommand{\calC}{\mathcal{C}}
\newcommand{\calD}{\mathcal{D}}
\newcommand{\calE}{\mathcal{E}}
\newcommand{\calF}{\mathcal{F}}
\newcommand{\calG}{\mathcal{G}}
\newcommand{\calH}{\mathcal{H}}
\newcommand{\calI}{\mathcal{I}}
\newcommand{\calJ}{\mathcal{J}}
\newcommand{\calK}{\mathcal{K}}
\newcommand{\calL}{\mathcal{L}}
\newcommand{\calM}{\mathcal{M}}
\newcommand{\calN}{\mathcal{N}}
\newcommand{\calO}{\mathcal{O}}
\newcommand{\calP}{\mathcal{P}}
\newcommand{\calQ}{\mathcal{Q}}
\newcommand{\calR}{\mathcal{R}}
\newcommand{\calS}{\mathcal{S}}
\newcommand{\calT}{\mathcal{T}}
\newcommand{\calU}{\mathcal{U}}
\newcommand{\calV}{\mathcal{V}}
\newcommand{\calW}{\mathcal{W}}
\newcommand{\calX}{\mathcal{X}}
\newcommand{\calY}{\mathcal{Y}}
\newcommand{\calZ}{\mathcal{Z}}

%sf font for transposition and spaces like R
\newcommand{\sfA}{\mathsf{A}}
\newcommand{\sfB}{\mathsf{B}}
\newcommand{\sfC}{\mathsf{C}}
\newcommand{\sfD}{\mathsf{D}}
\newcommand{\sfE}{\mathsf{E}}
\newcommand{\sfF}{\mathsf{F}}
\newcommand{\sfG}{\mathsf{G}}
\newcommand{\sfH}{\mathsf{H}}
\newcommand{\sfI}{\mathsf{I}}
\newcommand{\sfJ}{\mathsf{J}}
\newcommand{\sfK}{\mathsf{K}}
\newcommand{\sfL}{\mathsf{L}}
\newcommand{\sfM}{\mathsf{M}}
\newcommand{\sfN}{\mathsf{N}}
\newcommand{\sfO}{\mathsf{O}}
\newcommand{\sfP}{\mathsf{P}}
\newcommand{\sfQ}{\mathsf{Q}}
\newcommand{\sfR}{\mathsf{R}}
\newcommand{\sfS}{\mathsf{S}}
\newcommand{\sfT}{\mathsf{T}}
\newcommand{\sfU}{\mathsf{U}}
\newcommand{\sfV}{\mathsf{V}}
\newcommand{\sfW}{\mathsf{W}}
\newcommand{\sfX}{\mathsf{X}}
\newcommand{\sfY}{\mathsf{Y}}
\newcommand{\sfZ}{\mathsf{Z}}

\newcommand{\bt}{\begin{theorem}}
\newcommand{\et}{\end{theorem}}
\newcommand{\bl}{\begin{lemma}}
\newcommand{\el}{\end{lemma}}
\newcommand{\bp}{\begin{proposition}}
\newcommand{\ep}{\end{proposition}}
\newcommand{\bc}{\begin{corollary}}
\newcommand{\ec}{\end{corollary}}

\newcommand{\bd}{\begin{definition}\rm}
\newcommand{\ed}{\end{definition}}
\newcommand{\bex}{\begin{example}\rm}
\newcommand{\eex}{\end{example}}
\newcommand{\br}{\begin{remark}\rm}
\newcommand{\er}{\end{remark}}

\newcommand{\btbh}{\begin{table}[!ht]}
\newcommand{\etb}{\end{table}}
\newcommand{\bfgh}{\begin{figure}[!ht]}
\newcommand{\efg}{\end{figure}}

\newcommand{\bea}{\begin{eqnarray*}}
\newcommand{\eea}{\end{eqnarray*}}
\newcommand{\be}{\begin{eqnarray}}
\newcommand{\ee}{\end{eqnarray}}
%
\newcommand{\intl}{\int\limits}
\newcommand{\suml}{\sum\limits}
\newcommand{\liml}{\lim\limits}
\newcommand{\prodl}{\prod\limits}
\newcommand{\minl}{\min\limits}
\newcommand{\maxl}{\max\limits}
\newcommand{\supl}{\sup\limits}
%
\newcommand{\ve}{\varepsilon}
\newcommand{\vphi}{\varphi}
\newcommand{\ovl}{\overline}
\newcommand{\lm}{\lambda}
\def\wtilde{\widetilde}
\def\what{\widehat}

\newcommand{\ra}{\rightarrow}
\newcommand{\towith}[1]{\mathrel{\mathop{\longrightarrow}_{#1}}}

\def\bproof{\textbf{Proof.\ }}
\def\eproof{\hfill$\Box$\smallskip}

\def\spaceN{\mathsf{N}}
\def\spaceZ{\mathsf{Z}}
\def\spaceR{\mathsf{R}}
\def\spaceC{\mathsf{C}} %is not used?
\newcommand\Expect{\mathsf{E}}
%\newcommand\Variance{\mathsf{D}}

\newcommand{\bfw}{\mathbf{w}}

\def\last#1{{\underline{#1}}}
\def\llast#1{\underline{\underline{#1}}}
\def\first#1{{\mathstrut\overline{#1}}}
\def\ffirst#1{\mathstrut\overline{\mathstrut\overline{#1}}}
\def\overo#1{\overset{_\mathrm{o}}{#1}}
\newcommand{\ontop}[2]{\genfrac{}{}{0pt}{0}{#1}{#2}}
\def\bfpi{\mbox{\boldmath{$\pi$}}}
\def\bfmu{\mbox{\boldmath{$\mu$}}}
\def\bfPi{\mbox{\boldmath{$\Pi$}}}
\def\bfcR{\mbox{\boldmath{$\cR$}}}

\def\mmod{\mathop{\mathrm{mod}}}
\def\sspan{\mathop{\mathrm{span}}}
\def\rank{\mathop{\mathrm{rank}}}
\def\dist{\mathop{\mathrm{dist}}}

\newcommand{\reverse}{\mathop{\mathrm{rev}}}
\newcommand{\Arg}{\mathop\mathrm{Arg}}
\newcommand{\meas}{\mathop{\mathrm{meas}}}

\newcommand{\colspace}{\mathop{\mathrm{colspace}}}
\newcommand{\rowspace}{\mathop{\mathrm{rowspace}}}


\makeatletter
\def\adots{\mathinner{\mkern2mu\raise\p@\hbox{.}
\mkern2mu\raise4\p@\hbox{.}\mkern1mu
\raise7\p@\vbox{\kern7\p@\hbox{.}}\mkern1mu}}
\newcommand{\l@abcd}[2]{\hbox to\textwidth{#1\dotfill #2}}
\makeatother

\def\func{\mathop\mathrm}

% Some new definitions
\newcommand{\defeq}{\stackrel{def}{=}}
\newcommand{\frob}{\calF}
\def\trajmat#1{\calT_{\mathrm{#1}}}

\def\unit{\mathfrak{i}}


%\sectionfont{\centering}

%\subsectionfont{\centering}
%\subsubsectionfont{\normalsize}
%\setcounter{page}{1}

\begin{document}


\title{Исследование информационной матрицы в задаче оценивания сигнала конечного ранга}
\author{Н.К.~Звонарев}


%\address{Санкт-Петербургский государственный университет,\\ 
%	Российская Федерация, 199034, Санкт-Петербург, Университетская наб., 7/9}
% \email{nikitazvonarev@gmail.com}

\maketitle

Рассмотрим задачу оценивания сигнала $\tsS_N$ длины $N$ из зашумлённого сигнала $\tsX_N = \tsS_N + \tsY_N$, где $\tsS_N$ --- сигнал конечного ранга $r$, $\tsY_N$ --- подстрока стационарного процесса авторегрессии длины $N$. Ряд конечного ранга $\tsS_N = (s_0, \ldots, s_{N-1})$ порядка $r$ означает, что 
\begin{equation*}
s_n = \sum_{j = 1}^{r} a_j s_{n-j}, \quad n = r, \ldots, N - 1;\  a_r\neq 0.
\end{equation*}
Известно, что сигналы такого вида представимы в следующем параметрическом виде:
\begin{equation*} \label{task:parametricform}
s_n = \sum_j P_j(n) \exp(\alpha_j n) \cos(2 \pi \omega_j n + \psi_j),
\end{equation*}
где $P_j(n)$ --- многочлены от $n$. Ограничимся частным случаем, когда каждый многочлен представляет из себя константу, то есть
\begin{equation} \label{task:redparametricform}
s_n = \sum_j p_j \exp(\alpha_j n) \cos(2 \pi \omega_j n + \psi_j).
\end{equation}
Такие ряды могут быть записаны в комплексной форме в следующем виде:
\begin{equation} \label{task:compparametricform}
s_n = \sum_{j=1}^r c_j \mu_j^n = \sum_{j=1}^r c_j \exp(\lambda_j n),
\end{equation}
где $\mu_j = e^{\lambda_j}$, $\operatorname{Im}(\lambda_j) \in (-\pi; \pi]$, все $\mu_j$ различны, комплексные $\mu_i$ встречаются вместе со своим комплексным дополнением, при этом для $\mu_j = \overline \mu_k$: $c_j = \overline c_k$.

Задача данной работы --- исследовать границу Рао-Крамера для несмещённых оценок параметров сигнала, чтобы оценить снизу их порядок сходимости при увеличивающеся к бесконечности длине ряда $N$. В первой части работы будет приведён приём, позволяющий избавиться от сложной вещественной параметризации и перейти к более простой для расчётов комплексной. Во второй части работы получены порядки сходимости оценок констант $p_i$ в случае, когда все $\mu_j$ известны. В этом случае модель является линейной, и порядки сходимости оценок точные. В третьей части выведены границы снизу на порядки сходимости оценок констант $p_j$ и частот $\lambda_j$. В четвёртой части приведены необходимые для получения результата вспомогательные утверждения.

Получены следующие результаты: если ряд содержит экспоненциально-угасающую компоненту, для которой $|\mu_i| < 1$, то для неё не существует несмещённой состоятельной оценки ни для константы, ни для частоты. В случае, если известны частоты, то скорость сходимости констант равна $|\mu_j|^{-N}$, если $|\mu_j|>1$, и $N^{-1/2}$ для $|\mu_j|=1$. Если частоты неизвестны и $|\mu_j|=1$, то оценка снизу на скорость сходимости частоты равна $N^{-3/2}$, а константы --- $N^{-1/2}$.

\section{Вещественная и комплексная параметризация}
\subsection{Известные результаты}
В общем случае рассмотрим следующую задачу оценивания: дана выборка $X = M + \varepsilon$ длины $N$, где $\varepsilon$ --- многомерный случайный нормальный вектор с нулевым матожиданием и ковариационной матрицей $\Sigma_N$, $M = M(\Theta)$ --- детерминированный вектор средних, зависящий от параметра $\Theta \in \sfR^K$, в некоей окрестности $\Theta$ отображение $M(\Theta)$ является хотя бы один раз дифференцируемым. Требуется оценить параметр $\Theta = (\theta_1, \ldots, \theta_K)^\rmT$.

Определим информационную матрицу $\bfI = (\calI_{j, k}) \in \sfR^{K \times K}$, где элементы $\calI_{j, k} = \frac{\partial M^\rmT(\Theta)}{\partial \theta_j} \Sigma_N^{-1} \frac{\partial M(\Theta)}{\partial \theta_k}$. Заметим, что если определить скалярное произведение $\langle X, Y \rangle_{\Sigma^N} = X^\rmT \Sigma_N^{-1} Y$, то матрица $\bfI$ --- это матрица Грама для набора векторов $ \{\frac{\partial M(\Theta)}{\partial \theta_k}, k = 1,; \ldots, K\}$. Для удобства введём матрицу $\bfM \in \sfR^{N \times K}$, $\bfM = [\frac{\partial M(\Theta)}{\partial \theta_1}:\ldots:\frac{\partial M(\Theta)}{\partial \theta_K}]$.

В частном случае известная теорема Рао-Крамера утверждает следующее:
\begin{theorem} \label{th:raokramer}
	Рассмотрим несмещённую оценку $\widehat \Theta$ параметра $\Theta$. Тогда дисперсионная матрица $D \widehat \Theta \ge \bfI^{-1}$ в том смысле, что для любого $Z \in R^K$: $Z^T D \widehat \Theta Z \ge Z^T \bfI^{-1} Z$.
\end{theorem}

\subsection{Применение к задаче оценивания рядов}
В случае задачи оценивания сигнала \eqref{task:redparametricform}: $M = (s_0, \ldots, s_{N-1})$, вектор параметров $\Theta$ и его размерность зависит от выбора параметризации и набора известных данных. Во-первых, в данной работе рассматриваются два случая: когда частоты $\mu_j$ известны, и когда нет. В этих случаях $K = r$ и $K = 2r$ соответственно. Во-вторых, вид выбранной в \eqref{task:redparametricform} параметризации прямо зависит от того, лежит ли очередное $\mu_j$ на вещественной прямой или нет: если $\operatorname{Im}(\mu_j) = 0$, то соответствующие $\omega_j = 0$ и $\theta_j = 0$. Это не очень удобно, поэтому вместо этого рассматривается общая компексная параметризация \eqref{task:compparametricform}, в которой параметрами являются $c_k$ для случая известных частот, и $c_k$ вместе с $\lambda_k$ для неизвестных частот. 

Проблема же комплексной параметризации состоит в том, что для неё нельзя сформулировать результат, подобный \ref{th:raokramer}. Однако, можно заметить, что между вещественным и комплексным набором параметров существует непрерывное взаимнооднозначное в некоторой окрестности параметров $c_k$ и $\lambda_k$ отображение $Q:C^K \to C^K$ между вещественной параметризацией \eqref{task:redparametricform} и \eqref{task:compparametricform} (заметим, что вещественная параметризация \eqref{task:redparametricform} однозначно продолжается на комплексные числа). Невырожденный якобиан $\bfQ$ отображения $Q$ позволяет выразить матрицу $\bfM = \bfN \bfQ$, где $\bfN$ --- матрица частных производных по комплексным параметрам $c_k$ в первом и $c_k$ с $\lambda_k$ во втором случае: 
\begin{equation}\label{eq:Nwolambdas}
\bfN = [\frac{\partial \tsS^\rmT}{\partial c_1}:\ldots:\frac{\partial \tsS^\rmT}{\partial c_r}]
\end{equation}
и 
\begin{equation}\label{eq:N}
\bfN = [\frac{\partial \tsS^\rmT}{\partial c_1}:\ldots:\frac{\partial \tsS^\rmT}{\partial c_r}\frac{\partial \tsS^\rmT}{\partial \lambda_1}:\ldots:\frac{\partial \tsS^\rmT}{\partial \lambda_r}]
\end{equation}
соответственно. После этого можно ввести комплексный аналог информационной матрицы $\bfI$ --- матрицу $\bfJ$, определённую следующим образом:
\begin{equation}\label{eq:J}
\bfJ = \bfN^* \Sigma_N^{-1} \bfN,
\end{equation}
где под $\bfN^*$ обозначено эрмитово транспонирование матрицы $\bfN$. Удобно расширить определение скалярного произведения $\langle X, Y \rangle_{\Sigma_N}$ на комплексный случай как $\langle X, Y \rangle_{\Sigma^N} = X^* \Sigma_N^{-1} Y$, чтобы $\bfJ$ стала матрицей Грама столбцов матрицы $\bfN$.

Очевидно, что $\bfI = \bfQ^* \bfJ \bfQ$, и, таким образом, матрица $\bfJ$ невырождена тогда и только тогда, когда невырождена матрица $\bfI$, а порядок элементов в матрицах $\bfI^{-1}$ и $\bfJ^{-1}$ совпадает.

\section{Случай известных частот $\lambda_j$}
Для начала выявим вид матрицы $\bfN$ из \eqref{eq:Nwolambdas}.

\begin{proposition}
	Элементы матрицы $\bfN = (n_{j, k})$ имеют следующий вид:
	\begin{equation*}
	n_{j, k} = \mu_k^{j-1}.
	\end{equation*}
	Таким образом, матрица $\bfN$ --- комплексная матрица Вандермондта размера $N \times r$ для набора $\{\mu_1, \ldots, \mu_r\}$.
\end{proposition}
\begin{proof}
	Очевидно.
\end{proof}

Также обратим внимание на следующий факт.
\begin{remark}
	Модель данных \eqref{task:redparametricform} при известных $\lambda_k$ является линейной. Таким образом, полученная снизу граница на дисперсию оценки $\bfI^{-1}$, согласно теореме Гаусса-Маркова, являются точной для оценки по взвешенному методу наименьших квадратов: 
	\begin{equation*}
	\widehat \Theta = \bfI^{-1}\bfM^\rmT \Sigma_N^{-1} \tsX_N^\rmT  = \bfQ^{-1}\bfJ^{-1} \bfN^* \Sigma_N^{-1} \tsX_N^\rmT.
	\end{equation*}
\end{remark}

Вначале докажем результат, касающийся состоятельности этой оценки, для случая белого шума, а затем для общего случая процесса авторегрессии порядка $p$.

\begin{theorem} \label{th:wninconsistency}
	Пусть $\Sigma_N^{-1} = \frac{1}{\sigma^2}{\mathbf{1}}$, где $\mathbf{1}$ --- единичная матрица порядка $N \times N$, что соответствует случаю белого шума с дисперсией $\sigma^2$.
	
	Представим все $\mu_i$ в полярном виде: $\mu_j = \rho_j e^{i \varphi_j}$, $\varphi_j \in (-\pi, \pi]$. Пусть показатели экспонент упорядочены по невозрастанию модуля $\mu_j$, т.е. $\rho_j \ge \rho_{j+1}$. Показатели экспонент в этом случае будут разделены на три (возможно пустые) подстроки: те, у которых $|\mu_j|>1$, те, у которых $|\mu_j| = 1$ и те, у которых $|\mu_j| < 1$. Пусть соответствующее количество экспонент будет равно $r_A$ для $|\mu_j|>1$, $r_B$ для $|\mu_j|=1$ и $r_C$ для $|\mu_j|<1$. Тогда
	\begin{enumerate}
		\item Рассмотрим отнормированную матрицу $\widetilde \bfN = \bfN \bfL$, $\bfL = \text{diag}(l_1, \ldots, l_r)$, 
		\begin{equation*}
		l_j = \begin{cases}
		\frac{1}{\sqrt{\langle N_j, N_j \rangle_{\Sigma_N}}} e^{-i N \varphi_j}, & \rho_j > 1, \\
		\frac{1}{\sqrt{\langle N_j, N_j \rangle_{\Sigma_N}}}, & \rho_j \le 1,
		\end{cases}
		\end{equation*}
		$N_j$ --- $j$-й столбец матрицы $\bfN$.
		
		Тогда \begin{equation*}
		\lim_{N \to + \infty} \widetilde \bfN^* \Sigma_N^{-1} \widetilde \bfN = \left( \begin{array}{c|c|c}
		\widetilde \bfA & \mathbf{0} & \mathbf{0} \\ \hline
		\mathbf{0} & \mathbf{1}_{r_B} & \mathbf{0} \\ \hline
		\mathbf{0} & \mathbf{0} & \widetilde \bfC
		\end{array}  \right),
		\end{equation*}
		где матрица $\widetilde \bfA = (\tilde a_{j, k})$ --- невырожденная квадратная матрица порядка $r_A$, элементы которой равны
		\begin{equation*}
		\tilde a_{j, k} = \frac{\sqrt{\mu_j \overline{\mu_j} - 1}\sqrt{\mu_k \overline{\mu_k} - 1}}{\overline{\mu_j} \mu_k - 1},
		\end{equation*}
		$\mathbf{1}_{r_B}$ --- единичная квадратная матрица порядка $r_B$, $\widetilde \bfC = (\tilde c_{j, k})$ --- невырожденная квадратная матрица порядка $r_C$, элементы которой равны
		\begin{equation*}
		\tilde c_{j, k} = \frac{\sqrt{1 - \mu_{j+r_A+r_B} \overline{\mu_{j+r_A+r_B}}}\sqrt{1 - \mu_{k+r_A+r_B} \overline{\mu_{k+r_A+r_B}}}}{1 - \overline{\mu_{j+r_A+r_B}} \mu_{k+r_A+r_B}}.
	    \end{equation*}
	    
	    \item 
	    \begin{equation*}
	    \lim_{N \to + \infty} \bfJ^{-1} = \sigma^2 \left( \begin{array}{c|c}
	    \mathbf{0}_{r_A + r_B} & \mathbf{0} \\ \hline
	    \mathbf{0} & \bfC^{-1}
	    \end{array}  \right),
	    \end{equation*}
	    где $\mathbf{0}_{r_A + r_B}$ --- нулевая матрица порядка $r_A + r_B$, $\bfC = (c_{j,k})$ --- невырожденная квадратная матрица порядка $r_C$, элементы которой равны
	    \begin{equation*}
	    c_{j, k} = \frac{1}{1 - \overline{\mu_{j+r_A+r_B}} \mu_{k+r_A+r_B}}.
	    \end{equation*}
	\end{enumerate}
\end{theorem}
	
	\begin{proof}
		Не умаляя общности: пусть $\sigma^2 = 1$. Достаточно рассмотреть соответствующие пределы последовательностей. Это легко сделать, используя следующую формулу: $\sum_{j = 0}^{N-1} x^j = \frac{x^N-1}{x-1}$, если $x \ne 1$, и $N$ в противном случае.
		\begin{itemize}
			\item Пусть $\rho_j > 1$, $\rho_k > 1$. Тогда
			\begin{multline*}
			\sum_{t=0}^{N-1} \frac{\overline{\mu_j^t} \mu_k^t e^{i N (\varphi_j - \varphi_k)}}{\sqrt{\sum_{u=0}^{N-1} \mu_j^u \overline{\mu_j^u}} \sqrt{\sum_{u=0}^{N-1} \mu_k^u \overline{\mu_k^u}}} = \frac{(\rho_j \rho_k)^N - e^{i N(\varphi_j - \varphi_k)}}{\rho_j \rho_k e^{i (\varphi_k - \varphi_j)} - 1} \frac{\sqrt{\rho_j^2 - 1}}{\sqrt{\rho_j^{2N} - 1}} \cdot \\ \cdot \frac{\sqrt{\rho_k^2 - 1}}{\sqrt{\rho_k^{2N} - 1}} \to \tilde a_{j, k}.
			\end{multline*}
			
			\item Пусть $\rho_j > 1$, $\rho_k = 1$. Тогда
			\begin{equation*}
			\sum_{t=0}^{N-1} \frac{\overline{\mu_j^t} \mu_k^t e^{i N \varphi_j}}{\sqrt{\sum_{u=0}^{N-1} \mu_j^u \overline{\mu_j^u}} \sqrt{\sum_{u=0}^{N-1} \mu_k^u \overline{\mu_k^u}}} = \frac{(\rho_j )^N e^{i N \varphi_k} - e^{i N \varphi_j}}{\rho_j e^{i (\varphi_k - \varphi_j)} - 1} \frac{\sqrt{\rho_j^2 - 1}}{\sqrt{\rho_j^{2N} - 1}} \frac{1}{\sqrt{N}} \to 0.
			\end{equation*}
			
			\item Пусть $\rho_j > 1$, $\rho_k < 1$. Учитывая, что $\rho_j \rho_k < \rho_j$, получаем \footnote{Здесь нет случая, когда $\rho_k = 1/\rho_j$. Результат целиком аналогичен.}
			\begin{multline*}
			\sum_{t=0}^{N-1} \frac{\overline{\mu_j^t} \mu_k^t e^{i N \varphi_j}}{\sqrt{\sum_{u=0}^{N-1} \mu_j^u \overline{\mu_j^u}} \sqrt{\sum_{u=0}^{N-1} \mu_k^u \overline{\mu_k^u}}} = \frac{(\rho_j \rho_k)^N e^{i N \varphi_k} - e^{i N \varphi_j}}{\rho_j \rho_k e^{i (\varphi_k - \varphi_j)} - 1} \frac{\sqrt{\rho_j^2 - 1}}{\sqrt{\rho_j^{2N} - 1}} \cdot \\ \cdot \sqrt{\frac{1 - \rho_k^2}{1 - \rho_k^{2N}}} \to 0.
			\end{multline*}
			
			\item Пусть $\rho_j = 1$, $\rho_k = 1$, $j \ne k$. Тогда
			\begin{equation*}
			\sum_{t=0}^{N-1} \frac{\overline{\mu_j^t} \mu_k^t}{\sqrt{\sum_{u=0}^{N-1} \mu_j^u \overline{\mu_j^u}} \sqrt{\sum_{u=0}^{N-1} \mu_k^u \overline{\mu_k^u}}} = \frac{e^{i N (\varphi_k -\varphi_j)} - 1}{e^{i (\varphi_k - \varphi_j)} - 1} \frac{1}{\sqrt{N}} \frac{1}{\sqrt{N}} \to 0.
			\end{equation*}
			
			\item Пусть $\rho_j = 1$, $\rho_k < 1$. Тогда
			\begin{equation*}
			\sum_{t=0}^{N-1} \frac{\overline{\mu_j^t} \mu_k^t}{\sqrt{\sum_{u=0}^{N-1} \mu_j^u \overline{\mu_j^u}} \sqrt{\sum_{u=0}^{N-1} \mu_k^u \overline{\mu_k^u}}} = \frac{\rho_k^N e^{i N (\varphi_k -\varphi_j)} - 1}{\rho_k e^{i (\varphi_k - \varphi_j)} - 1}  \frac{1}{\sqrt{N}} \frac{\sqrt{1 - \rho_k^2}}{\sqrt{1 - \rho_k^{2N}}} \to 0.
			\end{equation*}
			
			\item Пусть $\rho_j < 1$, $\rho_k < 1$. Тогда
			\begin{multline*}
			\sum_{t=0}^{N-1} \frac{\overline{\mu_j^t} \mu_k^t}{\sqrt{\sum_{u=0}^{N-1} \mu_j^u \overline{\mu_j^u}} \sqrt{\sum_{u=0}^{N-1} \mu_k^u \overline{\mu_k^u}}} = \frac{1 - (\rho_j \rho_k)^N e^{i N (\varphi_k -\varphi_j)}}{1 - \rho_j \rho_k e^{i (\varphi_k - \varphi_j)}} \cdot \\ \cdot \frac{\sqrt{1 - \rho_j^2}}{\sqrt{1 - \rho_j^{2N}}} \frac{\sqrt{1 - \rho_k^2}}{\sqrt{1 - \rho_k^{2N}}} \to \frac{\sqrt{1 - \mu_{j} \overline{\mu_{j}}}\sqrt{1 - \mu_{k} \overline{\mu_{k}}}}{1 - \overline{\mu_{j}} \mu_{k}}.
			\end{multline*}
				
		\end{itemize}
		
		Матрицы $\widetilde \bfA$, $\widetilde \bfC$ и $\bfC$ --- Коши-подобные матрицы (Cauchy-like matrix), при этом все они являются невырожденными, см. \cite{Schechter1959}.
		
		Пункт 2 следует из пункта 1 теоремы с учётом применённой с помощью матрицы $\bfL$ нормировки.
	\end{proof}

Теперь докажем теорему в случае, когда шум представляет из себя стационарный процесс авторегрессии порядка $p$. Доказательство сводится к случаю белого шума \ref{th:wninconsistency} через предельное представление ковариационной матрицы.

Рассмотрим $\tsY = (y_t, \; t~\in~\sfN)$ --- стационарный гауссовский процесс авторегрессии  порядка $p$:
\begin{equation} \label{def:arp}
y_t = \sum_{j = 1}^p \beta_j y_{t - j} + \sigma \varepsilon_t,
\end{equation}
где $\beta_1, \ldots, \beta_p$ --- коэффициенты авторегрессии, $\sigma$ --- стандартное отклонение шума, $(\varepsilon_t, t \in \sfN)$ --- независимые нормально распределённые случайные величины с нулевым средним и единичной дисперсией.

Определим следующее: вектор $B = (1, -\beta_1, \ldots, -\beta_p)^\rmT \in \sfR^{p+1}$, $B = (b_1, \ldots, b_{p+1})^\rmT$ и комплексный полином $f(z) = 1 - \sum_{j=1}^p \beta_j z^j$, случайный вектор $\tsY_N = (y_1, \ldots, y_N)^\rmT$ длины $N \ge 2p + 1$ --- подстрока процесса $\tsY$ длины $N$ и $\Sigma_N$ --- его ковариационная матрица.

\begin{theorem} \label{th:arinconsistency}
	Пусть $\Sigma_N^{-1}$ --- обратная ковариационная матрица процесса случайного вектора $\tsY$.
	
	Представим все $\mu_i$ в полярном виде: $\mu_j = \rho_j e^{i \varphi_j}$, $\varphi_j \in (-\pi, \pi]$. Пусть показатели экспонент упорядочены по невозрастанию модуля $\mu_j$, т.е. $\rho_j \ge \rho_{j+1}$. Показатели экспонент в этом случае будут разделены на три (возможно пустые) подстроки: те, у которых $|\mu_j|>1$, те, у которых $|\mu_j| = 1$ и те, у которых $|\mu_j| < 1$. Пусть соответствующее количество экспонент будет равно $r_A$ для $|\mu_j|>1$, $r_B$ для $|\mu_j|=1$ и $r_C$ для $|\mu_j|<1$. Тогда
	\begin{enumerate}
		\item Рассмотрим отнормированную матрицу $\widetilde \bfN = \bfN \bfL$, $\bfL = \text{diag}(l_1, \ldots, l_r)$, 
		\begin{equation*}
		l_j = \begin{cases}
		\frac{1}{\sqrt{\langle N_j, N_j \rangle_{\Sigma_N}}} e^{-i N \varphi_j}, & \rho_j > 1, \\
		\frac{1}{\sqrt{\langle N_j, N_j \rangle_{\Sigma_N}}}, & \rho_j \le 1,
		\end{cases}
		\end{equation*}
		$N_j$ --- $j$-й столбец матрицы $\bfN$.
		
		Тогда \begin{equation*}
		\lim_{N \to + \infty} \widetilde \bfN^* \Sigma_N^{-1} \widetilde \bfN = \left( \begin{array}{c|c|c}
		\widetilde \bfA & \mathbf{0} & \mathbf{0} \\ \hline
		\mathbf{0} & \mathbf{1}_{r_B} & \mathbf{0} \\ \hline
		\mathbf{0} & \mathbf{0} & \widetilde \bfC
		\end{array}  \right),
		\end{equation*}
		где матрица $\widetilde \bfA = (\tilde a_{j, k})$ --- невырожденная квадратная матрица порядка $r_A$, элементы которой равны
		\begin{equation*}
		\tilde a_{j, k} = \frac{\sqrt{\mu_j \overline{\mu_j} - 1}\sqrt{\mu_k \overline{\mu_k} - 1} }{(\overline{\mu_j} \mu_k - 1)} \frac{\overline{f(1/\mu_j)} f(1/\mu_k) }{|f(1/\mu_j)| |f(1/\mu_k)|},
		\end{equation*}
		$\mathbf{1}_{r_B}$ --- единичная квадратная матрица порядка $r_B$, $\widetilde \bfC = (\tilde c_{j, k})$ --- невырожденная квадратная матрица порядка $r_C$, элементы которой равны
		\begin{multline*}
		\tilde c_{j, k} = \frac{\sqrt{1 - \mu_{j+r_A+r_B} \overline{\mu_{j+r_A+r_B}}}\sqrt{1 - \mu_{k+r_A+r_B} \overline{\mu_{k+r_A+r_B}}}}{1 - \overline{\mu_{j+r_A+r_B}} \mu_{k+r_A+r_B}} \cdot \\ \cdot \frac{\overline{f(\mu_{j+r_A+r_B})} f(\mu_{k+r_A+r_B})}{|f(\mu_{j+r_A+r_B})| |f(\mu_{k+r_A+r_B})|}.
		\end{multline*}
		
		\item 
	    \begin{equation*}
	    \lim_{N \to + \infty} \bfJ^{-1} = \sigma^2 \left( \begin{array}{c|c}
	    \mathbf{0}_{r_A + r_B} & \mathbf{0} \\ \hline
	    \mathbf{0} & \bfC^{-1}
	    \end{array}  \right),
	    \end{equation*}
	    где $\mathbf{0}_{r_A + r_B}$ --- нулевая матрица порядка $r_A + r_B$, $\bfC = (c_{j,k})$ --- невырожденная квадратная матрица порядка $r_C$, элементы которой равны
		\begin{equation*}
		c_{j, k} = \frac{\overline{f(\mu_{j + r_A + r_B})} f(\mu_{k + r_A + r_B})}{1 - \overline{\mu_{j+r_A+r_B}} \mu_{k+r_A+r_B}}.
		\end{equation*}
	\end{enumerate}
\end{theorem}

\begin{proof}
	Чтобы не пересчитывать заново все пределы, заметим следующее:
	\begin{itemize}
		\item Если $X$ и $Y$ --- два вектора, лежащие в $\sfZ^N$, то $\langle X, Y \rangle_{\Sigma_N} = X^* \Sigma_N^{-1} Y = \rev(X)^* \Sigma_N^{-1} \rev(Y)$, где $\rev(\cdot)$ --- оператор, разворачивающий вектор наоборот. Это следует из того, что $\Sigma_N^{-1}$ симметрична относительно побочной диагонали.
		\item Выполняется равенство $\langle X, Y \rangle_{\Sigma_N} = X^* \Sigma_N^{-1} Y = X^* \bfU_N^T \bfU_N Y$, где $\bfU_N$ --- верхнетреугольная матрица в разложении Холецкого матрицы $\Sigma_N^{-1}$. С достаточно большого $N$ можно применять результат теоремы \ref{th:invcov_chol}.
	\end{itemize}
	
	Пусть $X = (x_0, \ldots, x_{N - 1})^\rmT$, $X_t = \nu^t$, $Y = (y_0, \ldots, y_{N - 1})^\rmT$, $y_t = \xi^t$. Рассмотрим два случая (не умаляя общности, пусть $\sigma = 1$):
	\begin{itemize}
		\item $|\nu| > 1$. Тогда последовательности надо развернуть, получаем следующее:
		$X^* \Sigma_N^{-1} Y = \rev(X)^* \Sigma_N^{-1} \rev(Y) =  \rev(X)^* \bfU_N^T \bfU_N  \rev(Y) =$ \\$\sum_{t=0}^{N-{2p} -2} (\sum_{j=1}^{p+1}b_j \overline{\nu^{N - t - j}}) (\sum_{j=1}^{p+1}b_j \mu^{N - t - j}) + O(1) = $\\ $\sum_{t=0}^{N-{2p} - 2} (\overline{\nu^{N - t - 1}} \overline{f(1/\nu)})  (\mu^{N - t - 1}f(1/\mu)) + O(1) =$\\ $\overline{f(1/\nu)} f(1/\mu)\sum_{t=0}^{N-1} (\overline{\nu^{N - t - 1}} ) (\mu^{N - t - 1}) + O(1) = $\\
		$\overline{f(1/\nu)} f(1/\mu)\sum_{t=0}^{N-1} \overline{\nu^t} \mu^t + O(1)$.
		\item $|\nu| \le 1, |\mu| \le 1$. Последовательности разворачивать не нужно, имеем следующее:
		$X^* \Sigma_N^{-1} Y = X^* \bfU_N^T \bfU_N Y = \sum_{t=0}^{N-{2p}-2} (\sum_{j=1}^{p+1}b_j \overline{\nu^{
				t + j - 1}}) \cdot$ \\$\cdot (\sum_{j=1}^{p+1}b_j \mu^{t + j - 1}) + O((|\nu||\mu|)^{N - 2p - 1}) =$  
		$\sum_{t=0}^{N-{2p} -2} (\overline{\nu^t f(\nu)})(\mu^t f(\mu)) +$\\ $+ O((|\nu||\mu|)^{N})$ = 
		$\overline{f(\nu)} f(\mu) \sum_{t=0}^{N-1} \overline{\nu^t} \mu^t + O((|\nu||\mu|)^{N})$.
	\end{itemize}
	
	Далее доказательство аналогично теореме \ref{th:invcov_chol}. Заметим, что все знаменатели остаются корректными, и невырожденность соответствующих матриц сохраняется: $f(1/\mu_j)$ не может быть равно $0$ при $|\mu_j| > 1$, и $f(\mu_j)$ не может быть равно $0$ при $|\mu_j| \le 1$, потому что авторегрессионный шум является стационарным, а, значит, справедлива лемма \ref{th:stat}.
\end{proof}

\begin{remark}
	Теоремы \ref{th:wninconsistency} и \ref{th:arinconsistency} также дают оценку на скорость сходимости констант $c_j$ в \eqref{task:compparametricform} (или $p_j$ и $\omega_j$ в \eqref{task:redparametricform}): $O(\mu_j^{-N})$ для $|\mu_j| > 1$, $O(\frac{1}{\sqrt{N}})$ для $|\mu_j| = 1$ и $O(1)$ для $|\mu_j| < 1$.
\end{remark}


\section{Следствия из теорем \ref{th:wninconsistency} и \ref{th:arinconsistency}}
\begin{enumerate}
	\item Даже зная параметры шума и сигнальные корни, невозможно получить состоятельную оценку сигнала, если в сигнале есть убывающая экспонента или экспоненциально-модулированная синусоида. Тем более странно ожидать состоятельность, если параметры шума и сигнальные корни требуется оценивать. Тем не менее, результат не отвергает возможную (при реальном оценивании) состоятельность сигнала, если модули всех его сигнальных корней $|\mu_i| \ge 1$.
		
	\item Результат не зависит от того, стоит задача выделения из сигнала плюс белого шума или авторегрессионной стационарной последовательности конечного порядка $p$. \cite{ZhigljavskyGolyandinaGryaznov2016}
\end{enumerate}

\section{Сведения из теории авторегрессионных процессов}

\begin{lemma}\label{th:stat}
	Процесс авторегресии $\tsY$, определённый в \eqref{def:arp}, является стационарным тогда и только тогда, когда все корни полинома $f(z)$ по модулю строго больше $1$: $f(z) = 0$ в том и только в том случае, если $|z| > 1$.
\end{lemma}

\begin{theorem}\label{th:invcov}
	Рассмотрим случайный вектор $\tsY_N = (y_1, \ldots, y_N)^\rmT$ длины $N \ge 2p + 1$ --- подстрока процесса $\tsY$ длины $N$ и $\Sigma_N$ --- его ковариационную матрицу. Тогда обратная ковариационная матрица $\Sigma_N^{-1} = (s_{l, j; N})$ равна:
	\begin{equation*}
	s_{l, j; N} = \begin{cases}
	0, & |l - j| > p, \\
	\frac{1}{\sigma^2}\sum_{k = 1}^{\min(l, N + 1 - j, p + 1 - |l - j|)}b_k b_{k + |l - j|}, & l \le j, \\
	\frac{1}{\sigma^2}\sum_{k = 1}^{\min(j, N + 1 - l, p + 1 - |l - j|)}b_k b_{k + |l - j|}, & l > j.
	\end{cases}
	\end{equation*}
\end{theorem}
\begin{proof}
	Утверждение теоремы --- переписанный в другом виде результат, полученный в работе \cite{Verbyla1985}.
\end{proof}

\begin{theorem}\label{th:invcov_chol}
	В терминах теоремы \ref{th:invcov}: рассмотрим обратную ковариационную матрицу $\Sigma_N^{-1}$, где $N \ge 2p + 2$ и её разложение Холецкого $\Sigma_N^{-1} = \bfU^T_N \bfU_N$, $\bfU_N$ --- верхнетреугольная матрица порядка $N$. Тогда матрица $\bfU_N$ имеет следующий вид:
	\begin{equation*}
	\bfU_N = \frac{1}{\sigma} \begin{pmatrix}
	b_1 & b_2 & \ldots & b_{p+1} &  &  &  &  \\ 
	& b_1 & b_2 & \ldots & b_{p+1} &  &  &  \\ 
	&  & \ddots & \ddots & \ddots & \ddots &  &  \\ 
	&  &  & b_1 & b_2 & \ldots & b_{p+1} &  \\ 
	&  &  &  & \multicolumn{4}{c}{\multirow{4}{*}{$\bfU_{2p+1}$}} \\
	&  &  &  & \multicolumn{4}{c}{\multirow{4}{*}{}} \\
	&  &  &  & \multicolumn{4}{c}{\multirow{4}{*}{}} \\
	&  &  &  & \multicolumn{4}{c}{\multirow{4}{*}{}}	
	\end{pmatrix},
	\end{equation*}
	где количество строчек в верхнем блоке равно $N - (2p + 1)$, $\bfU_{2p+1}$ --- разложение Холецкого матрицы $\Sigma_{2p+1}^{-1}$: $\Sigma_{2p+1}^{-1} = \bfU_{2p+1}^\rmT \bfU_{2p+1}$.
\end{theorem}
\begin{proof}
	Заметим, что разложение Холецкого для матрицы $\bfU_N$, $N \ge 2p + 2$, можно представить, как
	\begin{equation*}
	\bfU_N = \frac{1}{\sigma} \begin{pmatrix}
	b_1 & b_2 & \ldots & b_{p+1} &  \\ 
	& \multicolumn{4}{c}{\multirow{4}{*}{$\bfU_{N - 1}$}} \\
	& \multicolumn{4}{c}{\multirow{4}{*}{}} \\
	& \multicolumn{4}{c}{\multirow{4}{*}{}} \\
	& \multicolumn{4}{c}{\multirow{4}{*}{}}	
	\end{pmatrix},
	\end{equation*}	
	где $\bfU_{N-1}$ --- разложение Холецкого для матрицы $\Sigma_{N-1}^{-1}$. Это следствие вида матрицы $\Sigma_{N}$ из теоремы \ref{th:invcov} и алгоритма вычисления разложения Холецкого. Далее доказательство продолжается по индукции. \cite{stoica2005spectral}
\end{proof}

\bibliographystyle{gost705}
\bibliography{zvonarev}
\end{document}
