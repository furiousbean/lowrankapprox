\documentclass[12pt,a4paper]{article}

\usepackage[text={14cm,20cm}]{geometry}
\usepackage{amsmath,amssymb,amsthm,amscd,amsfonts}
\usepackage[utf8x]{inputenc}

\usepackage[T2A]{fontenc}
\usepackage[russian]{babel}


\usepackage{euscript}
\usepackage{relsize}
\usepackage{mathdots}
\usepackage{graphicx}
%\usepackage{epstopdf}
\usepackage{indentfirst}
\usepackage{empheq}
\usepackage{multirow}

\newcommand*\widefbox[1]{\fbox{\hspace{2em}#1\hspace{2em}}}

\usepackage[colorlinks, urlcolor=blue, pdfborder={0 0 0 [0 0]}]{hyperref}

\hyphenation{Struc-tu-red}
\hyphenation{Ran-do-mized}
\hyphenation{Ma-xi-mi-za-tion}
\DeclareMathOperator*{\argmax}{arg\,max}
\DeclareMathOperator*{\argmin}{arg\,min}
\DeclareMathOperator{\tr}{tr}
\providecommand*{\BibDash}{}

\def\rank{\mathop{\mathrm{rank}}}
\def\rev{\mathop{\mathrm{rev}}}
\DeclareSymbolFont{bbold}{U}{bbold}{m}{n}
\DeclareSymbolFontAlphabet{\mathbbold}{bbold}

\newtheorem{corollary}{Следствие}
\newtheorem{proposition}{Предложение}
\newtheorem{algorithm}{Алгоритм}
\newtheorem{theorem}{Теорема}
\newtheorem{lemma}{Лемма}
\newtheorem{remark}{Замечание}
\newtheorem{problem}{Задача}

%\usepackage{euscript}

%new calligraphic font for subspaces
\usepackage{euscript}
\newcommand{\spA}{\EuScript{A}}
\newcommand{\spB}{\EuScript{B}}
\newcommand{\spC}{\EuScript{C}}
\newcommand{\spD}{\EuScript{D}}
\newcommand{\spE}{\EuScript{E}}
\newcommand{\spF}{\EuScript{F}}
\newcommand{\spG}{\EuScript{G}}
\newcommand{\spH}{\EuScript{H}}
\newcommand{\spI}{\EuScript{I}}
\newcommand{\spJ}{\EuScript{J}}
\newcommand{\spK}{\EuScript{K}}
\newcommand{\spL}{\EuScript{L}}
\newcommand{\spM}{\EuScript{M}}
\newcommand{\spN}{\EuScript{N}}
\newcommand{\spO}{\EuScript{O}}
\newcommand{\spP}{\EuScript{P}}
\newcommand{\spQ}{\EuScript{Q}}
\newcommand{\spR}{\EuScript{R}}
\newcommand{\spS}{\EuScript{S}}
\newcommand{\spT}{\EuScript{T}}
\newcommand{\spU}{\EuScript{U}}
\newcommand{\spV}{\EuScript{V}}
\newcommand{\spW}{\EuScript{W}}
\newcommand{\spX}{\EuScript{X}}
\newcommand{\spY}{\EuScript{Y}}
\newcommand{\spZ}{\EuScript{Z}}

%font for text indices like transposition X^\mathrm{T}
\newcommand{\rmA}{\mathrm{A}}
\newcommand{\rmB}{\mathrm{B}}
\newcommand{\rmC}{\mathrm{C}}
\newcommand{\rmD}{\mathrm{D}}
\newcommand{\rmE}{\mathrm{E}}
\newcommand{\rmF}{\mathrm{F}}
\newcommand{\rmG}{\mathrm{G}}
\newcommand{\rmH}{\mathrm{H}}
\newcommand{\rmI}{\mathrm{I}}
\newcommand{\rmJ}{\mathrm{J}}
\newcommand{\rmK}{\mathrm{K}}
\newcommand{\rmL}{\mathrm{L}}
\newcommand{\rmM}{\mathrm{M}}
\newcommand{\rmN}{\mathrm{N}}
\newcommand{\rmO}{\mathrm{O}}
\newcommand{\rmP}{\mathrm{P}}
\newcommand{\rmQ}{\mathrm{Q}}
\newcommand{\rmR}{\mathrm{R}}
\newcommand{\rmS}{\mathrm{S}}
\newcommand{\rmT}{\mathrm{T}}
\newcommand{\rmU}{\mathrm{U}}
\newcommand{\rmV}{\mathrm{V}}
\newcommand{\rmW}{\mathrm{W}}
\newcommand{\rmX}{\mathrm{X}}
\newcommand{\rmY}{\mathrm{Y}}
\newcommand{\rmZ}{\mathrm{Z}}

%tt font for time series
\newcommand{\tsA}{\mathbb{A}}
\newcommand{\tsB}{\mathbb{B}}
\newcommand{\tsC}{\mathbb{C}}
\newcommand{\tsD}{\mathbb{D}}
\newcommand{\tsE}{\mathbb{E}}
\newcommand{\tsF}{\mathbb{F}}
\newcommand{\tsG}{\mathbb{G}}
\newcommand{\tsH}{\mathbb{H}}
\newcommand{\tsI}{\mathbb{I}}
\newcommand{\tsJ}{\mathbb{J}}
\newcommand{\tsK}{\mathbb{K}}
\newcommand{\tsL}{\mathbb{L}}
\newcommand{\tsM}{\mathbb{M}}
\newcommand{\tsN}{\mathbb{N}}
\newcommand{\tsO}{\mathbb{O}}
\newcommand{\tsP}{\mathbb{P}}
\newcommand{\tsQ}{\mathbb{Q}}
\newcommand{\tsR}{\mathbb{R}}
\newcommand{\tsS}{\mathbb{S}}
\newcommand{\tsT}{\mathbb{T}}
\newcommand{\tsU}{\mathbb{U}}
\newcommand{\tsV}{\mathbb{V}}
\newcommand{\tsW}{\mathbb{W}}
\newcommand{\tsX}{\mathbb{X}}
\newcommand{\tsY}{\mathbb{Y}}
\newcommand{\tsZ}{\mathbb{Z}}

%bf font for matrices
\newcommand{\bfA}{\mathbf{A}}
\newcommand{\bfB}{\mathbf{B}}
\newcommand{\bfC}{\mathbf{C}}
\newcommand{\bfD}{\mathbf{D}}
\newcommand{\bfE}{\mathbf{E}}
\newcommand{\bfF}{\mathbf{F}}
\newcommand{\bfG}{\mathbf{G}}
\newcommand{\bfH}{\mathbf{H}}
\newcommand{\bfI}{\mathbf{I}}
\newcommand{\bfJ}{\mathbf{J}}
\newcommand{\bfK}{\mathbf{K}}
\newcommand{\bfL}{\mathbf{L}}
\newcommand{\bfM}{\mathbf{M}}
\newcommand{\bfN}{\mathbf{N}}
\newcommand{\bfO}{\mathbf{O}}
\newcommand{\bfP}{\mathbf{P}}
\newcommand{\bfQ}{\mathbf{Q}}
\newcommand{\bfR}{\mathbf{R}}
\newcommand{\bfS}{\mathbf{S}}
\newcommand{\bfT}{\mathbf{T}}
\newcommand{\bfU}{\mathbf{U}}
\newcommand{\bfV}{\mathbf{V}}
\newcommand{\bfW}{\mathbf{W}}
\newcommand{\bfX}{\mathbf{X}}
\newcommand{\bfY}{\mathbf{Y}}
\newcommand{\bfZ}{\mathbf{Z}}

%bb font for standard spaces and expectation
\newcommand{\bbA}{\mathbb{A}}
\newcommand{\bbB}{\mathbb{B}}
\newcommand{\bbC}{\mathbb{C}}
\newcommand{\bbD}{\mathbb{D}}
\newcommand{\bbE}{\mathbb{E}}
\newcommand{\bbF}{\mathbb{F}}
\newcommand{\bbG}{\mathbb{G}}
\newcommand{\bbH}{\mathbb{H}}
\newcommand{\bbI}{\mathbb{I}}
\newcommand{\bbJ}{\mathbb{J}}
\newcommand{\bbK}{\mathbb{K}}
\newcommand{\bbL}{\mathbb{L}}
\newcommand{\bbM}{\mathbb{M}}
\newcommand{\bbN}{\mathbb{N}}
\newcommand{\bbO}{\mathbb{O}}
\newcommand{\bbP}{\mathbb{P}}
\newcommand{\bbQ}{\mathbb{Q}}
\newcommand{\bbR}{\mathbb{R}}
\newcommand{\bbS}{\mathbb{S}}
\newcommand{\bbT}{\mathbb{T}}
\newcommand{\bbU}{\mathbb{U}}
\newcommand{\bbV}{\mathbb{V}}
\newcommand{\bbW}{\mathbb{W}}
\newcommand{\bbX}{\mathbb{X}}
\newcommand{\bbY}{\mathbb{Y}}
\newcommand{\bbZ}{\mathbb{Z}}

%got font for any case
\newcommand{\gA}{\mathfrak{A}}
\newcommand{\gB}{\mathfrak{B}}
\newcommand{\gC}{\mathfrak{C}}
\newcommand{\gD}{\mathfrak{D}}
\newcommand{\gE}{\mathfrak{E}}
\newcommand{\gF}{\mathfrak{F}}
\newcommand{\gG}{\mathfrak{G}}
\newcommand{\gH}{\mathfrak{H}}
\newcommand{\gI}{\mathfrak{I}}
\newcommand{\gJ}{\mathfrak{J}}
\newcommand{\gK}{\mathfrak{K}}
\newcommand{\gL}{\mathfrak{L}}
\newcommand{\gM}{\mathfrak{M}}
\newcommand{\gN}{\mathfrak{N}}
\newcommand{\gO}{\mathfrak{O}}
\newcommand{\gP}{\mathfrak{P}}
\newcommand{\gQ}{\mathfrak{Q}}
\newcommand{\gR}{\mathfrak{R}}
\newcommand{\gS}{\mathfrak{S}}
\newcommand{\gT}{\mathfrak{T}}
\newcommand{\gU}{\mathfrak{U}}
\newcommand{\gV}{\mathfrak{V}}
\newcommand{\gW}{\mathfrak{W}}
\newcommand{\gX}{\mathfrak{X}}
\newcommand{\gY}{\mathfrak{Y}}
\newcommand{\gZ}{\mathfrak{Z}}

%old calligraphic font
\newcommand{\calA}{\mathcal{A}}
\newcommand{\calB}{\mathcal{B}}
\newcommand{\calC}{\mathcal{C}}
\newcommand{\calD}{\mathcal{D}}
\newcommand{\calE}{\mathcal{E}}
\newcommand{\calF}{\mathcal{F}}
\newcommand{\calG}{\mathcal{G}}
\newcommand{\calH}{\mathcal{H}}
\newcommand{\calI}{\mathcal{I}}
\newcommand{\calJ}{\mathcal{J}}
\newcommand{\calK}{\mathcal{K}}
\newcommand{\calL}{\mathcal{L}}
\newcommand{\calM}{\mathcal{M}}
\newcommand{\calN}{\mathcal{N}}
\newcommand{\calO}{\mathcal{O}}
\newcommand{\calP}{\mathcal{P}}
\newcommand{\calQ}{\mathcal{Q}}
\newcommand{\calR}{\mathcal{R}}
\newcommand{\calS}{\mathcal{S}}
\newcommand{\calT}{\mathcal{T}}
\newcommand{\calU}{\mathcal{U}}
\newcommand{\calV}{\mathcal{V}}
\newcommand{\calW}{\mathcal{W}}
\newcommand{\calX}{\mathcal{X}}
\newcommand{\calY}{\mathcal{Y}}
\newcommand{\calZ}{\mathcal{Z}}

%sf font for transposition and spaces like R
\newcommand{\sfA}{\mathsf{A}}
\newcommand{\sfB}{\mathsf{B}}
\newcommand{\sfC}{\mathsf{C}}
\newcommand{\sfD}{\mathsf{D}}
\newcommand{\sfE}{\mathsf{E}}
\newcommand{\sfF}{\mathsf{F}}
\newcommand{\sfG}{\mathsf{G}}
\newcommand{\sfH}{\mathsf{H}}
\newcommand{\sfI}{\mathsf{I}}
\newcommand{\sfJ}{\mathsf{J}}
\newcommand{\sfK}{\mathsf{K}}
\newcommand{\sfL}{\mathsf{L}}
\newcommand{\sfM}{\mathsf{M}}
\newcommand{\sfN}{\mathsf{N}}
\newcommand{\sfO}{\mathsf{O}}
\newcommand{\sfP}{\mathsf{P}}
\newcommand{\sfQ}{\mathsf{Q}}
\newcommand{\sfR}{\mathsf{R}}
\newcommand{\sfS}{\mathsf{S}}
\newcommand{\sfT}{\mathsf{T}}
\newcommand{\sfU}{\mathsf{U}}
\newcommand{\sfV}{\mathsf{V}}
\newcommand{\sfW}{\mathsf{W}}
\newcommand{\sfX}{\mathsf{X}}
\newcommand{\sfY}{\mathsf{Y}}
\newcommand{\sfZ}{\mathsf{Z}}

\newcommand{\bt}{\begin{theorem}}
\newcommand{\et}{\end{theorem}}
\newcommand{\bl}{\begin{lemma}}
\newcommand{\el}{\end{lemma}}
\newcommand{\bp}{\begin{proposition}}
\newcommand{\ep}{\end{proposition}}
\newcommand{\bc}{\begin{corollary}}
\newcommand{\ec}{\end{corollary}}

\newcommand{\bd}{\begin{definition}\rm}
\newcommand{\ed}{\end{definition}}
\newcommand{\bex}{\begin{example}\rm}
\newcommand{\eex}{\end{example}}
\newcommand{\br}{\begin{remark}\rm}
\newcommand{\er}{\end{remark}}

\newcommand{\btbh}{\begin{table}[!ht]}
\newcommand{\etb}{\end{table}}
\newcommand{\bfgh}{\begin{figure}[!ht]}
\newcommand{\efg}{\end{figure}}

\newcommand{\bea}{\begin{eqnarray*}}
\newcommand{\eea}{\end{eqnarray*}}
\newcommand{\be}{\begin{eqnarray}}
\newcommand{\ee}{\end{eqnarray}}
%
\newcommand{\intl}{\int\limits}
\newcommand{\suml}{\sum\limits}
\newcommand{\liml}{\lim\limits}
\newcommand{\prodl}{\prod\limits}
\newcommand{\minl}{\min\limits}
\newcommand{\maxl}{\max\limits}
\newcommand{\supl}{\sup\limits}
%
\newcommand{\ve}{\varepsilon}
\newcommand{\vphi}{\varphi}
\newcommand{\ovl}{\overline}
\newcommand{\lm}{\lambda}
\def\wtilde{\widetilde}
\def\what{\widehat}

\newcommand{\ra}{\rightarrow}
\newcommand{\towith}[1]{\mathrel{\mathop{\longrightarrow}_{#1}}}

\def\bproof{\textbf{Proof.\ }}
\def\eproof{\hfill$\Box$\smallskip}

\def\spaceN{\mathsf{N}}
\def\spaceZ{\mathsf{Z}}
\def\spaceR{\mathsf{R}}
\def\spaceC{\mathsf{C}} %is not used?
\newcommand\Expect{\mathsf{E}}
%\newcommand\Variance{\mathsf{D}}

\newcommand{\bfw}{\mathbf{w}}

\def\last#1{{\underline{#1}}}
\def\llast#1{\underline{\underline{#1}}}
\def\first#1{{\mathstrut\overline{#1}}}
\def\ffirst#1{\mathstrut\overline{\mathstrut\overline{#1}}}
\def\overo#1{\overset{_\mathrm{o}}{#1}}
\newcommand{\ontop}[2]{\genfrac{}{}{0pt}{0}{#1}{#2}}
\def\bfpi{\mbox{\boldmath{$\pi$}}}
\def\bfmu{\mbox{\boldmath{$\mu$}}}
\def\bfPi{\mbox{\boldmath{$\Pi$}}}
\def\bfcR{\mbox{\boldmath{$\cR$}}}

\def\mmod{\mathop{\mathrm{mod}}}
\def\sspan{\mathop{\mathrm{span}}}
\def\rank{\mathop{\mathrm{rank}}}
\def\dist{\mathop{\mathrm{dist}}}

\newcommand{\reverse}{\mathop{\mathrm{rev}}}
\newcommand{\Arg}{\mathop\mathrm{Arg}}
\newcommand{\meas}{\mathop{\mathrm{meas}}}

\newcommand{\colspace}{\mathop{\mathrm{colspace}}}
\newcommand{\rowspace}{\mathop{\mathrm{rowspace}}}


\makeatletter
\def\adots{\mathinner{\mkern2mu\raise\p@\hbox{.}
\mkern2mu\raise4\p@\hbox{.}\mkern1mu
\raise7\p@\vbox{\kern7\p@\hbox{.}}\mkern1mu}}
\newcommand{\l@abcd}[2]{\hbox to\textwidth{#1\dotfill #2}}
\makeatother

\def\func{\mathop\mathrm}

% Some new definitions
\newcommand{\defeq}{\stackrel{def}{=}}
\newcommand{\frob}{\calF}
\def\trajmat#1{\calT_{\mathrm{#1}}}

\def\unit{\mathfrak{i}}


%\sectionfont{\centering}

%\subsectionfont{\centering}
%\subsubsectionfont{\normalsize}
%\setcounter{page}{1}

\begin{document}


\title{Несостоятельность регрессии на экспоненциальные временные ряды}
\author{Н.К.~Звонарев}


%\address{Санкт-Петербургский государственный университет,\\ 
%	Российская Федерация, 199034, Санкт-Петербург, Универси етская наб., 7/9}
% \email{nikitazvonarev@gmail.com}

\maketitle

\section{Сведения из теории авторегрессионных процессов}

Рассмотрим стационарный гауссовский процесс авторегрессии $\tsY = (y_t, t~\in~\sfN)$ порядка $p$:
\begin{equation} \label{def:arp}
y_t = \sum_{i = 1}^p \beta_i y_{t - i} + \varepsilon_t,
\end{equation}
где $\beta_1, \ldots, \beta_p$ --- коэффициенты авторегрессии, $(\varepsilon_t, t \in \sfN)$ --- независимые нормально распределённые случайные величины с нулевым средним и дисперсией $\sigma^2$.

Определим следующие вещи: вектор $B = (b_1, \ldots, b_{p+1})^\rmT \in \sfR^{p+1}$, $B = (1, -\beta_1, \ldots, -\beta_p)^\rmT$, и комплексный полином $f(z) = 1 - \sum_{i=1}^p \beta_i z^i$\footnote{Под $i$ я буду обозначать натуральный индекс, комплексную единицу я буду обозначать как $I$.}.

\begin{lemma}\label{th:stat}
	Процесс авторегресии $\tsY$ является стационарным тогда и только тогда, когда все корни полинома $f(z)$ по модулю строго больше $1$: когда $f(z) = 0$, то $|z| > 1$.
\end{lemma}

\begin{theorem}\label{th:invcov}
	Рассмотрим случайный вектор $\tsY_N = (y_1, \ldots, y_N)^\rmT$ длины $N \ge 2p + 1$ --- кусок процесса $\tsY$ длины $N$, $\Sigma_N$ --- его ковариационная матрица. Тогда обратная ковариационная матрица $\Sigma_N^{-1} = (s_{i, j, N})$ равна:
	\begin{equation*}
	s_{i, j, N} = \begin{cases}
	0, & |i - j| > p, \\
	\frac{1}{\sigma^2}\sum_{k = 1}^{\min(i, N + 1 - j, p + 1 - |i - j|)}b_k b_{k + |i - j|}, & i \le j, \\
	\frac{1}{\sigma^2}\sum_{k = 1}^{\min(j, N + 1 - i, p + 1 - |i - j|)}b_k b_{k + |i - j|}, & i > j.
	\end{cases}
	\end{equation*}
\end{theorem}
\begin{proof}
	Утверждение теоремы --- переписанный в другом виде результат, полученный в работе \cite{Verbyla1985}.
\end{proof}

\begin{theorem}\label{th:invcov_chol}
	Не умаляя общности: пусть $\sigma = 1$. Рассмотрим обратную ковариационную матрицу $\Sigma_N^{-1}$, где $N \ge 2p + 2$ и её разложение Холецкого $\Sigma_N^{-1} = \bfU^T_N \bfU_N$, $\bfU_N$ --- верхнетреугольная матрица порядка $N$. Тогда матрица $\bfU_N$ имеет следующий вид:
	\begin{equation*}
	\bfU_N = \begin{pmatrix}
	b_1 & b_2 & \ldots & b_{p+1} &  &  &  &  \\ 
	& b_1 & b_2 & \ldots & b_{p+1} &  &  &  \\ 
	&  & \ddots & \ddots & \ddots & \ddots &  &  \\ 
	&  &  & b_1 & b_2 & \ldots & b_{p+1} &  \\ 
	&  &  &  & \multicolumn{4}{c}{\multirow{4}{*}{$\bfU_{2p+1}$}} \\
	&  &  &  & \multicolumn{4}{c}{\multirow{4}{*}{}} \\
	&  &  &  & \multicolumn{4}{c}{\multirow{4}{*}{}} \\
	&  &  &  & \multicolumn{4}{c}{\multirow{4}{*}{}}	
	\end{pmatrix},
	\end{equation*}
	где количество строчек в верхнем блоке равно $N - (2p + 1)$, $\bfU_{2p+1}$ --- разложение Холецкого матрицы $\Sigma_{2p+1}^{-1}$: $\Sigma_{2p+1}^{-1} = \bfU_{2p+1}^\rmT \bfU_{2p+1}$.
\end{theorem}
\begin{proof}
	Заметим, что разложение Холецкого для матрицы $\bfU_N$, $N \ge 2p + 2$, можно представить, как
	\begin{equation*}
	\bfU_N = \begin{pmatrix}
	b_1 & b_2 & \ldots & b_{p+1} &  \\ 
	& \multicolumn{4}{c}{\multirow{4}{*}{$\bfU_{N - 1}$}} \\
	& \multicolumn{4}{c}{\multirow{4}{*}{}} \\
	& \multicolumn{4}{c}{\multirow{4}{*}{}} \\
	& \multicolumn{4}{c}{\multirow{4}{*}{}}	
	\end{pmatrix},
	\end{equation*}	
	где $\bfU_{N-1}$ --- разложение Холецкого для матрицы $\Sigma_{N-1}^{-1}$. Это следствие вида матрицы $\Sigma_{N}$ из теоремы \ref{th:invcov} и алгоритма вычисления разложения Холецкого. Далее продолжаем по индукции.
\end{proof}

\section{Линейная регрессия на экспоненциальные временные ряды}
Рассмотрим задачу оценивания сигнала, состоящего из $r$ комплексных экспонент длины $N$: пусть нам дан ряд $\tsX_N = (x_t, t = 0, \ldots, N-1)$: $\tsX_n = \tsS_N + \tsY_N$, $x_t = \sum_{i=1}^r c_i \mu_i^t + y_t$, где $\mu_i \ne 0$ --- попарно различные показатели экспонент (в данном случае, для простоты рассматриваются комплексные экспоненты, чтобы упростить случаи синусоидальных и экспоненциально-модулированных синусоидальных сигналов), $y_t$ --- шум, представляющий из себя стационарный процесс авторегрессии. При $p = 0$ имеем белый шум. Параметры шума (коэффициенты $\beta_1$, \ldots, $\beta_p$) и показатели экспонент известны, что является упрощением реальной задачи. Требуется дать оценку параметрам $C = (c_1, \ldots, c_r)^\rmT$. Тогда оценку $\widehat C_N$ коэффициентов $C$ можно получить по методу наименьших квадратов:
\begin{equation*}
\widehat C_N = (\bfM_N^* \Sigma_N^{-1} \bfM_N)^{-1}\bfM_N^* \Sigma_N^{-1} \tsX_N^\rmT,
\end{equation*}
где $\bfM_N$ --- матрица регрессоров порядка $N \times r$, $\bfM_N = (m_{i, j})$, $m_{i, j} = \mu_j^{i- 1}$, $\Sigma_N$~---~матрица ковариаций случайного вектора $\tsY_N$, под $\bfM_N^*$ определено эрмитово транспонирование матрицы $\bfM_N$.

Согласно теореме Гаусса-Маркова, такая оценка является наилучшей в классе несмещённых линейных оценок $C$. Матрица ковариаций $\Sigma_{\widehat C_N}$ вектора $\widehat C_N$ при этом равна
\begin{equation*}
\Sigma_{\widehat C_N} = (\bfM_N^* \Sigma_N^{-1} \bfM_N)^{-1}.
\end{equation*}
В случае белого шума, при $p = 0$ имеем $\Sigma_{\widehat C_N} = \sigma^2 (\bfM_N^* \bfM_N)^{-1}$. Учитывая, что оценка является несмещённой, её состоятельность выполняется тогда и только тогда, когда $ \lim_{N \to +\infty} \Sigma_{\widehat C_N} = \mathbf{0}$.

Вначале докажем результат, касающийся состоятельности этой оценки, для случая белого шума, а затем для общего случая процесса авторегрессии порядка $p$.

\begin{theorem} \label{th:wninconsistency}
	Представим все $\mu_i$ в полярном виде: $\mu_i = \rho_i e^{I \varphi_i}$, $\varphi_i \in (-\pi, \pi]$. Пусть показатели экспонент упорядочены по невозрастанию модуля $\mu_i$, т.е. $\rho_i \ge \rho_{i+1}$. Показатели экспонент в этом случае будут разделены на три (возможно пустые) подпоследовательности: те, у которых $|\mu_i|>1$, те, у которых $|\mu_i| = 1$ и те, у которых $|\mu_i| < 1$. Пусть соответствующее количество экспонент будет равно $r_A$ для $|\mu_i|>1$, $r_B$ для $|\mu_i|>1$ и $r_C$ для $|\mu_i|<1$. Тогда
	\begin{enumerate}
		\item Рассмотрим отнормированную матрицу $\widetilde \bfM_N = (\tilde m_{i, j})$: 
		\begin{equation*}
		\tilde m_{i, j} = \begin{cases}
		\frac{\mu_j^{i-1}}{\sqrt{\sum_{k=0}^{N-1} \mu_j^k \overline{\mu_j^k}}} e^{-I N \varphi_j}, & \rho_j > 1, \\
		\frac{\mu_j^{i-1}}{\sqrt{\sum_{k=0}^{N-1} \mu_j^k \overline{\mu_j^k}}}, & \rho_j \le 1.
		\end{cases}
		\end{equation*}
		
		Тогда \begin{equation*}
		\lim_{N \to + \infty} \widetilde \bfM_N^* \widetilde \bfM_N = \left( \begin{array}{c|c|c}
		\widetilde \bfA & \mathbf{0} & \mathbf{0} \\ \hline
		\mathbf{0} & \bfI & \mathbf{0} \\ \hline
		\mathbf{0} & \mathbf{0} & \widetilde \bfC
		\end{array}  \right),
		\end{equation*}
		где матрица $\widetilde \bfA = (\tilde a_{i, j})$ --- невырожденная квадратная матрица порядка $r_A$, элементы которой равны
		\begin{equation*}
		\tilde a_{i, j} = \frac{\sqrt{\mu_i \overline{\mu_i} - 1}\sqrt{\mu_j \overline{\mu_j} - 1}}{\overline{\mu_i} \mu_j - 1},
		\end{equation*}
		$\bfI$ --- единичная квадратная матрица порядка $r_B$, $\widetilde \bfC = (\tilde c_{i, j})$ --- невырожденная квадратная матрица порядка $r_C$, элементы которой равны
		\begin{equation*}
		\tilde c_{i, j} = \frac{\sqrt{1 - \mu_{i+r_A+r_B} \overline{\mu_{i+r_A+r_B}}}\sqrt{1 - \mu_{j+r_A+r_B} \overline{\mu_{j+r_A+r_B}}}}{1 - \overline{\mu_{i+r_A+r_B}} \mu_{j+r_A+r_B}}.
	    \end{equation*}
	    
	    \item 
	    \begin{equation*}
	    \lim_{N \to + \infty} \sigma^2 (\bfM_N^* \bfM_N)^{-1} = \sigma^2 \left( \begin{array}{c|c}
	    \mathbf{0}_{r_A + r_B} & \mathbf{0} \\ \hline
	    \mathbf{0} & \bfC^{-1}
	    \end{array}  \right),
	    \end{equation*}
	    где $\mathbf{0}_{r_A + r_B}$ --- нулевая матрица порядка $r_A + r_B$, $\bfC = (c_{i,j})$ --- невырожденная квадратная матрица порядка $r_C$, элементы которой равны
	    \begin{equation*}
	    c_{i, j} = \frac{1}{1 - \overline{\mu_{i+r_A+r_B}} \mu_{j+r_A+r_B}}.
	    \end{equation*}
	\end{enumerate}
\end{theorem}
	
	\begin{proof}
		Достаточно рассмотреть соответствующие пределы последовательностей. Для этого требуется использовать формулу $\sum_{i = 0}^{N-1} x^i = \frac{x^N-1}{x-1}$, если $x \ne 1$, и $N$ в противном случае.
		\begin{itemize}
			\item Пусть $\rho_i > 1$, $\rho_j > 1$. Тогда
			\begin{multline*}
			\sum_{t=0}^{N-1} \frac{\overline{\mu_i^t} \mu_j^t e^{I N (\varphi_i - \varphi_j)}}{\sqrt{\sum_{k=0}^{N-1} \mu_i^k \overline{\mu_i^k}} \sqrt{\sum_{k=0}^{N-1} \mu_j^k \overline{\mu_j^k}}} = \frac{(\rho_i \rho_j)^N - e^{I N(\varphi_i - \varphi_j)}}{\rho_i \rho_j e^{I (\varphi_j - \varphi_i)} - 1} \frac{\sqrt{\rho_i^2 - 1}}{\sqrt{\rho_i^{2N} - 1}} \cdot \\ \cdot \frac{\sqrt{\rho_j^2 - 1}}{\sqrt{\rho_j^{2N} - 1}} \to \tilde a_{i, j}.
			\end{multline*}
			
			\item Пусть $\rho_i > 1$, $\rho_j = 1$. Тогда
			\begin{equation*}
			\sum_{t=0}^{N-1} \frac{\overline{\mu_i^t} \mu_j^t e^{I N \varphi_i}}{\sqrt{\sum_{k=0}^{N-1} \mu_i^k \overline{\mu_i^k}} \sqrt{\sum_{k=0}^{N-1} \mu_j^k \overline{\mu_j^k}}} = \frac{(\rho_i )^N e^{I N \varphi_j} - e^{I N \varphi_i}}{\rho_i e^{I (\varphi_j - \varphi_i)} - 1} \frac{\sqrt{\rho_i^2 - 1}}{\sqrt{\rho_i^{2N} - 1}} \frac{1}{\sqrt{N}} \to 0.
			\end{equation*}
			
			\item Пусть $\rho_i > 1$, $\rho_j < 1$. Учитывая, что $\rho_i \rho_j < \rho_i$, получаем \footnote{Здесь нет случая, когда $\rho_j = 1/\rho_i$. Результат целиком аналогичен.}
			\begin{multline*}
			\sum_{t=0}^{N-1} \frac{\overline{\mu_i^t} \mu_j^t e^{I N \varphi_i}}{\sqrt{\sum_{k=0}^{N-1} \mu_i^k \overline{\mu_i^k}} \sqrt{\sum_{k=0}^{N-1} \mu_j^k \overline{\mu_j^k}}} = \frac{(\rho_i \rho_j)^N e^{I N \varphi_j} - e^{I N \varphi_i}}{\rho_i \rho_j e^{I (\varphi_j - \varphi_i)} - 1} \frac{\sqrt{\rho_i^2 - 1}}{\sqrt{\rho_i^{2N} - 1}} \cdot \\ \cdot \sqrt{\frac{1 - \rho_j^2}{1 - \rho_j^{2N}}} \to 0.
			\end{multline*}
			
			\item Пусть $\rho_i = 1$, $\rho_j = 1$, $i \ne j$. Тогда
			\begin{equation*}
			\sum_{t=0}^{N-1} \frac{\overline{\mu_i^t} \mu_j^t}{\sqrt{\sum_{k=0}^{N-1} \mu_i^k \overline{\mu_i^k}} \sqrt{\sum_{k=0}^{N-1} \mu_j^k \overline{\mu_j^k}}} = \frac{e^{I N (\varphi_j -\varphi_i)} - 1}{e^{I (\varphi_j - \varphi_i)} - 1} \frac{1}{\sqrt{N}} \frac{1}{\sqrt{N}} \to 0.
			\end{equation*}
			
			\item Пусть $\rho_i = 1$, $\rho_j < 1$. Тогда
			\begin{equation*}
			\sum_{t=0}^{N-1} \frac{\overline{\mu_i^t} \mu_j^t}{\sqrt{\sum_{k=0}^{N-1} \mu_i^k \overline{\mu_i^k}} \sqrt{\sum_{k=0}^{N-1} \mu_j^k \overline{\mu_j^k}}} = \frac{\rho_j^N e^{I N (\varphi_j -\varphi_i)} - 1}{\rho_j e^{I (\varphi_j - \varphi_i)} - 1}  \frac{1}{\sqrt{N}} \frac{\sqrt{1 - \rho_j^2}}{\sqrt{1 - \rho_j^{2N}}} \to 0.
			\end{equation*}
			
			\item Пусть $\rho_i < 1$, $\rho_j < 1$. Тогда
			\begin{multline*}
			\sum_{t=0}^{N-1} \frac{\overline{\mu_i^t} \mu_j^t}{\sqrt{\sum_{k=0}^{N-1} \mu_i^k \overline{\mu_i^k}} \sqrt{\sum_{k=0}^{N-1} \mu_j^k \overline{\mu_j^k}}} = \frac{1 - (\rho_i \rho_j)^N e^{I N (\varphi_j -\varphi_i)}}{1 - \rho_i \rho_j e^{I (\varphi_j - \varphi_i)}} \cdot \\ \cdot \frac{\sqrt{1 - \rho_i^2}}{\sqrt{1 - \rho_i^{2N}}} \frac{\sqrt{1 - \rho_j^2}}{\sqrt{1 - \rho_j^{2N}}} \to \frac{\sqrt{1 - \mu_{i} \overline{\mu_{i}}}\sqrt{1 - \mu_{j} \overline{\mu_{j}}}}{1 - \overline{\mu_{i}} \mu_{j}}.
			\end{multline*}
				
		\end{itemize}
		
		Матрицы $\widetilde \bfA$, $\widetilde \bfC$ и $\bfC$ --- Коши-подобные матрицы (Cauchy-like matrix), при этом все они невырожденные, см. \cite{Schechter1959}.
		
		Пункт 2 следует из пункта 1 теоремы с учётом применённой нормировки.
	\end{proof}

Теперь докажем теорему в общем виде, сведя вычисление пределов к случаю белого шума.

\begin{theorem} \label{th:arinconsistency}
	Представим все $\mu_i$ в полярном виде: $\mu_i = \rho_i e^{I \varphi_i}$, $\varphi_i \in (-\pi, \pi]$. Пусть показатели экспонент упорядочены по невозрастанию модуля $\mu_i$, т.е. $\rho_i \ge \rho_{i+1}$. Показатели экспонент в этом случае будут разделены на три (возможно пустые) подпоследовательности: те, у которых $|\mu_i|>1$, те, у которых $|\mu_i| = 1$ и те, у которых $|\mu_i| < 1$. Пусть соответствующее количество экспонент будет равно $r_A$ для $|\mu_i|>1$, $r_B$ для $|\mu_i|>1$ и $r_C$ для $|\mu_i|<1$. Тогда
	\begin{enumerate}
		\item Рассмотрим отнормированную матрицу $\widetilde \bfM_N = (\tilde m_{i, j})$: 
		\begin{equation*}
		\tilde m_{i, j} = \begin{cases}
		\frac{\mu_j^{i-1}}{\sqrt{\sum_{k=0}^{N-1} \mu_j^k \overline{\mu_j^k}}} e^{-I N \varphi_j}, & \rho_j > 1, \\
		\frac{\mu_j^{i-1}}{\sqrt{\sum_{k=0}^{N-1} \mu_j^k \overline{\mu_j^k}}}, & \rho_j \le 1.
		\end{cases}
		\end{equation*}
		
		Тогда \begin{equation*}
		\lim_{N \to + \infty} \widetilde \bfM_N^* \Sigma_N^{-1} \widetilde \bfM_N = \left( \begin{array}{c|c|c}
		\widetilde \bfA & \mathbf{0} & \mathbf{0} \\ \hline
		\mathbf{0} & \bfI / \sigma^2& \mathbf{0} \\ \hline
		\mathbf{0} & \mathbf{0} & \widetilde \bfC
		\end{array}  \right),
		\end{equation*}
		где матрица $\widetilde \bfA = (\tilde a_{i, j})$ --- невырожденная квадратная матрица порядка $r_A$, элементы которой равны
		\begin{equation*}
		\tilde a_{i, j} = \frac{1}{\sigma^2} \frac{\sqrt{\mu_i \overline{\mu_i} - 1}\sqrt{\mu_j \overline{\mu_j} - 1} \overline{f(1/\mu_i)} f(1/\mu_j) }{(\overline{\mu_i} \mu_j - 1) |f(1/\mu_i)| |f(1/\mu_j)|},
		\end{equation*}
		$\bfI$ --- единичная квадратная матрица порядка $r_B$, $\widetilde \bfC = (\tilde c_{i, j})$ --- невырожденная квадратная матрица порядка $r_C$, элементы которой равны
		\begin{multline*}
		\tilde c_{i, j} = \frac{1}{\sigma^2} \frac{\sqrt{1 - \mu_{i+r_A+r_B} \overline{\mu_{i+r_A+r_B}}}\sqrt{1 - \mu_{j+r_A+r_B} \overline{\mu_{j+r_A+r_B}}}}{1 - \overline{\mu_{i+r_A+r_B}} \mu_{j+r_A+r_B}} \cdot \\ \cdot \frac{\overline{f(\mu_{i+r_A+r_B})} f(\mu_{j+r_A+r_B})}{|f(\mu_{i+r_A+r_B})| |f(\mu_{j+r_A+r_B})|}.
		\end{multline*}
		
		\item 
		\begin{equation*}
		\lim_{N \to + \infty} (\bfM_N^* \Sigma_N^{-1} \bfM_N)^{-1} = \sigma^2 \left( \begin{array}{c|c}
		\mathbf{0}_{r_A + r_B} & \mathbf{0} \\ \hline
		\mathbf{0} & \bfC^{-1}
		\end{array}  \right),
		\end{equation*}
		где $\mathbf{0}_{r_A + r_B}$ --- нулевая матрица порядка $r_A + r_B$, $\bfC = (c_{i,j})$ --- невырожденная квадратная матрица порядка $r_C$, элементы которой равны
		\begin{equation*}
		c_{i, j} = \frac{\overline{f(\mu_{i + r_A + r_B})} f(\mu_{j + r_A + r_B})}{1 - \overline{\mu_{i+r_A+r_B}} \mu_{j+r_A+r_B}}.
		\end{equation*}
	\end{enumerate}
\end{theorem}

\begin{proof}
	Чтобы не пересчитывать заново все пределы, заметим следующее:
	\begin{itemize}
		\item Если $A$ и $B$ --- два вектора, лежащие в $\sfZ^N$, то $A^* \Sigma_N^{-1} B = \rev(A)^* \Sigma_N^{-1} \rev(B)$, где $\rev(\cdot)$ --- оператор, разворачивающий вектор наоборот. Следует из того, что $\Sigma_N^{-1}$ симметрична относительно побочной диагонали.
		\item Выполняется равенство $A^* \Sigma_N^{-1} B = A^* \bfU_N^T \bfU_N B$, где $\bfU_N$ --- верхнетреугольная матрица в разложении Холецкого матрицы $\Sigma_N^{-1}$. С достаточно большого $N$ можно применять результат теоремы \ref{th:invcov_chol}.
	\end{itemize}
	
	Пусть $A = (a_0, \ldots, a_{N - 1})^\rmT$, $a_i = \nu^i$, $B = (b_0, \ldots, b_{N - 1})^\rmT$, $b_i = \xi^i$. Рассмотрим два случая (не умаляя общности, пусть $\sigma = 1$):
	\begin{itemize}
		\item $|\nu| > 1$. Тогда последовательности надо развернуть, получаем следующее:
		$A^* \Sigma_N^{-1} B = \rev(A)^* \Sigma_N^{-1} \rev(B) =  \rev(A)^* \bfU_N^T \bfU_N  \rev(B) =$ \\$\sum_{i=0}^{N-{2p} -2} (\sum_{j=1}^{p+1}b_j \overline{\nu^{N - i - j}}) (\sum_{j=1}^{p+1}b_j \mu^{N - i - j}) + O(1) = $\\ $\sum_{i=0}^{N-{2p} - 2} (\overline{\nu^{N - i - 1}} \overline{f(1/\nu)})  (\mu^{N - i - 1}f(1/\mu)) + O(1) =$\\ $\overline{f(1/\nu)} f(1/\mu)\sum_{i=0}^{N-1} (\overline{\nu^{N - i - 1}} ) (\mu^{N - i - 1}) + O(1) = $\\
		$\overline{f(1/\nu)} f(1/\mu)\sum_{i=0}^{N-1} \overline{\nu^i} \mu^i + O(1)$.
		\item $|\nu| \le 1, |\mu| \le 1$. Последовательности разворачивать не нужно, имеем следующее:
		$A^* \Sigma_N^{-1} B = A^* \bfU_N^T \bfU_N B = \sum_{i=0}^{N-{2p}-2} (\sum_{j=1}^{p+1}b_j \overline{\nu^{
				i + j - 1}}) \cdot$ \\$\cdot (\sum_{j=1}^{p+1}b_j \mu^{i + j - 1}) + O((|\nu||\mu|)^{N - 2p - 1}) =$  
		$\sum_{i=0}^{N-{2p} -2} (\overline{\nu^i f(\nu)})(\mu^i f(\mu)) +$\\ $+ O((|\nu||\mu|)^{N})$ = 
		$\overline{f(\nu)} f(\mu) \sum_{i=0}^{N-1} \overline{\nu^i} \mu^i + O((|\nu||\mu|)^{N})$.
	\end{itemize}
	
	Далее доказательство аналогично теореме \ref{th:invcov_chol}. Заметим, что все знаменатели остаются корректными, и невырожденность соответствующих матриц сохраняется: $f(1/\mu_i)$ не может быть равно $0$ при $|\mu_i| > 1$, и $f(\mu_i)$ не может быть равно $0$ при $|\mu_i| \le 1$, потому что авторегрессионный шум является стационарным, а, значит, справедлива лемма \ref{th:stat}.
\end{proof}

\section{Следствия из теорем \ref{th:wninconsistency} и \ref{th:arinconsistency}}
\begin{enumerate}
	\item Даже зная параметры шума и сигнальные корни, невозможно получить состоятельную оценку сигнала, если в сигнале есть убывающая экспонента или экспоненциально-модулированная синусоида. Тем более странно ожидать состоятельность, если параметры шума и сигнальные корни требуется оценивать. Тем не менее, результат не отвергает возможную (при реальном оценивании) состоятельность сигнала, если модули всех его сигнальных корней $|\mu_i| \ge 1$.
	
	\item Теоремы также дают оценки снизу на скорость сходимости при реальном оценивании: $O(\mu_i^{-N})$ для $|\mu_i| > 1$, $O(\frac{1}{\sqrt{N}})$ для $|\mu_i| = 1$ и $O(1)$ для $|\mu_i| < 1$. Быстрее нереально.
	
	\item Результат не зависит от того, стоит задача выделения из сигнала плюс белого шума или авторегрессионной стационарной последовательности конечного порядка $p$.
\end{enumerate}
\bibliographystyle{gost705}
\bibliography{zvonarev}
\end{document}
