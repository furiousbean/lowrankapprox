
\documentclass[12pt,a4paper]{article}
\usepackage[a4paper,
mag=1000, includefoot,
left=3cm, right=1.5cm, top=2cm, bottom=2cm, headsep=1cm, footskip=1cm]{geometry}
\usepackage{mathtext}
\usepackage{cmap}
\usepackage[utf8x]{inputenc}
\usepackage[russian]{babel}
\usepackage[T2A]{fontenc}
\usepackage{amsmath,amssymb,amsthm,amscd,amsfonts}
\usepackage{euscript}
\usepackage{relsize}
\usepackage{mathdots}
\usepackage{graphicx}
\usepackage{epstopdf}
\usepackage{caption2}
\usepackage{indentfirst}
\usepackage{fancyhdr}
\usepackage{sectsty}
\usepackage{titlesec}
\usepackage{mathtext}%русские буквы в формулах

\usepackage[colorlinks, urlcolor=blue, pdfborder={0 0 0 [0 0]}]{hyperref}

\hyphenation{Struc-tu-red}
\hyphenation{Ran-do-mized}
\hyphenation{Ma-xi-mi-za-tion}
\DeclareMathOperator*{\argmax}{arg\,max}
\DeclareMathOperator*{\argmin}{arg\,min}
\DeclareMathOperator{\tr}{tr}
\providecommand*{\BibDash}{}

\def\rank{\mathop{\mathrm{rank}}}
\newtheorem{corollary}{Следствие}
\newtheorem{proposition}{Предложение}
\newtheorem{algorithm}{Алгоритм}
\newtheorem{lemma}{Лемма}

\usepackage{euscript}
\newcommand{\bt}{\begin{theorem}}
\newcommand{\et}{\end{theorem}}
\newcommand{\bl}{\begin{lemma}}
\newcommand{\el}{\end{lemma}}
\newcommand{\bp}{\begin{proposition}}
\newcommand{\ep}{\end{proposition}}
\newcommand{\bc}{\begin{corollary}}
\newcommand{\ec}{\end{corollary}}

\newcommand{\bd}{\begin{definition}\rm}
\newcommand{\ed}{\end{definition}}
\newcommand{\bex}{\begin{example}\rm}
\newcommand{\eex}{\end{example}}
\newcommand{\br}{\begin{remark}\rm}
\newcommand{\er}{\end{remark}}

\newcommand{\btbh}{\begin{table}[!ht]}
\newcommand{\etb}{\end{table}}
\newcommand{\bfgh}{\begin{figure}[!ht]}
\newcommand{\efg}{\end{figure}}

\newcommand{\bea}{\begin{eqnarray*}}
\newcommand{\eea}{\end{eqnarray*}}
\newcommand{\be}{\begin{eqnarray}}
\newcommand{\ee}{\end{eqnarray}}
%
\newcommand{\intl}{\int\limits}
\newcommand{\suml}{\sum\limits}
\newcommand{\liml}{\lim\limits}
\newcommand{\prodl}{\prod\limits}
\newcommand{\minl}{\min\limits}
\newcommand{\maxl}{\max\limits}
\newcommand{\supl}{\sup\limits}
%
\newcommand{\ve}{\varepsilon}
\newcommand{\vphi}{\varphi}
\newcommand{\ovl}{\overline}
\newcommand{\lm}{\lambda}
\def\wtilde{\widetilde}
\def\what{\widehat}

\newcommand{\ra}{\rightarrow}
\newcommand{\towith}[1]{\mathrel{\mathop{\longrightarrow}_{#1}}}

\def\bproof{\textbf{Proof.\ }}
\def\eproof{\hfill$\Box$\smallskip}

\def\spaceN{\mathsf{N}}
\def\spaceZ{\mathsf{Z}}
\def\spaceR{\mathsf{R}}
\def\spaceC{\mathsf{C}} %is not used?
\newcommand\Expect{\mathsf{E}}
%\newcommand\Variance{\mathsf{D}}

\newcommand{\bfw}{\mathbf{w}}

\def\last#1{{\underline{#1}}}
\def\llast#1{\underline{\underline{#1}}}
\def\first#1{{\mathstrut\overline{#1}}}
\def\ffirst#1{\mathstrut\overline{\mathstrut\overline{#1}}}
\def\overo#1{\overset{_\mathrm{o}}{#1}}
\newcommand{\ontop}[2]{\genfrac{}{}{0pt}{0}{#1}{#2}}
\def\bfpi{\mbox{\boldmath{$\pi$}}}
\def\bfmu{\mbox{\boldmath{$\mu$}}}
\def\bfPi{\mbox{\boldmath{$\Pi$}}}
\def\bfcR{\mbox{\boldmath{$\cR$}}}

\def\mmod{\mathop{\mathrm{mod}}}
\def\sspan{\mathop{\mathrm{span}}}
\def\rank{\mathop{\mathrm{rank}}}
\def\dist{\mathop{\mathrm{dist}}}

\newcommand{\reverse}{\mathop{\mathrm{rev}}}
\newcommand{\Arg}{\mathop\mathrm{Arg}}
\newcommand{\meas}{\mathop{\mathrm{meas}}}

\newcommand{\colspace}{\mathop{\mathrm{colspace}}}
\newcommand{\rowspace}{\mathop{\mathrm{rowspace}}}


\makeatletter
\def\adots{\mathinner{\mkern2mu\raise\p@\hbox{.}
\mkern2mu\raise4\p@\hbox{.}\mkern1mu
\raise7\p@\vbox{\kern7\p@\hbox{.}}\mkern1mu}}
\newcommand{\l@abcd}[2]{\hbox to\textwidth{#1\dotfill #2}}
\makeatother

\def\func{\mathop\mathrm}

% Some new definitions
\newcommand{\defeq}{\stackrel{def}{=}}
\newcommand{\frob}{\calF}
\def\trajmat#1{\calT_{\mathrm{#1}}}

\def\unit{\mathfrak{i}}


%new calligraphic font for subspaces
\usepackage{euscript}
\newcommand{\spA}{\EuScript{A}}
\newcommand{\spB}{\EuScript{B}}
\newcommand{\spC}{\EuScript{C}}
\newcommand{\spD}{\EuScript{D}}
\newcommand{\spE}{\EuScript{E}}
\newcommand{\spF}{\EuScript{F}}
\newcommand{\spG}{\EuScript{G}}
\newcommand{\spH}{\EuScript{H}}
\newcommand{\spI}{\EuScript{I}}
\newcommand{\spJ}{\EuScript{J}}
\newcommand{\spK}{\EuScript{K}}
\newcommand{\spL}{\EuScript{L}}
\newcommand{\spM}{\EuScript{M}}
\newcommand{\spN}{\EuScript{N}}
\newcommand{\spO}{\EuScript{O}}
\newcommand{\spP}{\EuScript{P}}
\newcommand{\spQ}{\EuScript{Q}}
\newcommand{\spR}{\EuScript{R}}
\newcommand{\spS}{\EuScript{S}}
\newcommand{\spT}{\EuScript{T}}
\newcommand{\spU}{\EuScript{U}}
\newcommand{\spV}{\EuScript{V}}
\newcommand{\spW}{\EuScript{W}}
\newcommand{\spX}{\EuScript{X}}
\newcommand{\spY}{\EuScript{Y}}
\newcommand{\spZ}{\EuScript{Z}}

%font for text indices like transposition X^\mathrm{T}
\newcommand{\rmA}{\mathrm{A}}
\newcommand{\rmB}{\mathrm{B}}
\newcommand{\rmC}{\mathrm{C}}
\newcommand{\rmD}{\mathrm{D}}
\newcommand{\rmE}{\mathrm{E}}
\newcommand{\rmF}{\mathrm{F}}
\newcommand{\rmG}{\mathrm{G}}
\newcommand{\rmH}{\mathrm{H}}
\newcommand{\rmI}{\mathrm{I}}
\newcommand{\rmJ}{\mathrm{J}}
\newcommand{\rmK}{\mathrm{K}}
\newcommand{\rmL}{\mathrm{L}}
\newcommand{\rmM}{\mathrm{M}}
\newcommand{\rmN}{\mathrm{N}}
\newcommand{\rmO}{\mathrm{O}}
\newcommand{\rmP}{\mathrm{P}}
\newcommand{\rmQ}{\mathrm{Q}}
\newcommand{\rmR}{\mathrm{R}}
\newcommand{\rmS}{\mathrm{S}}
\newcommand{\rmT}{\mathrm{T}}
\newcommand{\rmU}{\mathrm{U}}
\newcommand{\rmV}{\mathrm{V}}
\newcommand{\rmW}{\mathrm{W}}
\newcommand{\rmX}{\mathrm{X}}
\newcommand{\rmY}{\mathrm{Y}}
\newcommand{\rmZ}{\mathrm{Z}}

%tt font for time series
\newcommand{\tsA}{\mathbb{A}}
\newcommand{\tsB}{\mathbb{B}}
\newcommand{\tsC}{\mathbb{C}}
\newcommand{\tsD}{\mathbb{D}}
\newcommand{\tsE}{\mathbb{E}}
\newcommand{\tsF}{\mathbb{F}}
\newcommand{\tsG}{\mathbb{G}}
\newcommand{\tsH}{\mathbb{H}}
\newcommand{\tsI}{\mathbb{I}}
\newcommand{\tsJ}{\mathbb{J}}
\newcommand{\tsK}{\mathbb{K}}
\newcommand{\tsL}{\mathbb{L}}
\newcommand{\tsM}{\mathbb{M}}
\newcommand{\tsN}{\mathbb{N}}
\newcommand{\tsO}{\mathbb{O}}
\newcommand{\tsP}{\mathbb{P}}
\newcommand{\tsQ}{\mathbb{Q}}
\newcommand{\tsR}{\mathbb{R}}
\newcommand{\tsS}{\mathbb{S}}
\newcommand{\tsT}{\mathbb{T}}
\newcommand{\tsU}{\mathbb{U}}
\newcommand{\tsV}{\mathbb{V}}
\newcommand{\tsW}{\mathbb{W}}
\newcommand{\tsX}{\mathbb{X}}
\newcommand{\tsY}{\mathbb{Y}}
\newcommand{\tsZ}{\mathbb{Z}}

%bf font for matrices
\newcommand{\bfA}{\mathbf{A}}
\newcommand{\bfB}{\mathbf{B}}
\newcommand{\bfC}{\mathbf{C}}
\newcommand{\bfD}{\mathbf{D}}
\newcommand{\bfE}{\mathbf{E}}
\newcommand{\bfF}{\mathbf{F}}
\newcommand{\bfG}{\mathbf{G}}
\newcommand{\bfH}{\mathbf{H}}
\newcommand{\bfI}{\mathbf{I}}
\newcommand{\bfJ}{\mathbf{J}}
\newcommand{\bfK}{\mathbf{K}}
\newcommand{\bfL}{\mathbf{L}}
\newcommand{\bfM}{\mathbf{M}}
\newcommand{\bfN}{\mathbf{N}}
\newcommand{\bfO}{\mathbf{O}}
\newcommand{\bfP}{\mathbf{P}}
\newcommand{\bfQ}{\mathbf{Q}}
\newcommand{\bfR}{\mathbf{R}}
\newcommand{\bfS}{\mathbf{S}}
\newcommand{\bfT}{\mathbf{T}}
\newcommand{\bfU}{\mathbf{U}}
\newcommand{\bfV}{\mathbf{V}}
\newcommand{\bfW}{\mathbf{W}}
\newcommand{\bfX}{\mathbf{X}}
\newcommand{\bfY}{\mathbf{Y}}
\newcommand{\bfZ}{\mathbf{Z}}

%bb font for standard spaces and expectation
\newcommand{\bbA}{\mathbb{A}}
\newcommand{\bbB}{\mathbb{B}}
\newcommand{\bbC}{\mathbb{C}}
\newcommand{\bbD}{\mathbb{D}}
\newcommand{\bbE}{\mathbb{E}}
\newcommand{\bbF}{\mathbb{F}}
\newcommand{\bbG}{\mathbb{G}}
\newcommand{\bbH}{\mathbb{H}}
\newcommand{\bbI}{\mathbb{I}}
\newcommand{\bbJ}{\mathbb{J}}
\newcommand{\bbK}{\mathbb{K}}
\newcommand{\bbL}{\mathbb{L}}
\newcommand{\bbM}{\mathbb{M}}
\newcommand{\bbN}{\mathbb{N}}
\newcommand{\bbO}{\mathbb{O}}
\newcommand{\bbP}{\mathbb{P}}
\newcommand{\bbQ}{\mathbb{Q}}
\newcommand{\bbR}{\mathbb{R}}
\newcommand{\bbS}{\mathbb{S}}
\newcommand{\bbT}{\mathbb{T}}
\newcommand{\bbU}{\mathbb{U}}
\newcommand{\bbV}{\mathbb{V}}
\newcommand{\bbW}{\mathbb{W}}
\newcommand{\bbX}{\mathbb{X}}
\newcommand{\bbY}{\mathbb{Y}}
\newcommand{\bbZ}{\mathbb{Z}}

%got font for any case
\newcommand{\gA}{\mathfrak{A}}
\newcommand{\gB}{\mathfrak{B}}
\newcommand{\gC}{\mathfrak{C}}
\newcommand{\gD}{\mathfrak{D}}
\newcommand{\gE}{\mathfrak{E}}
\newcommand{\gF}{\mathfrak{F}}
\newcommand{\gG}{\mathfrak{G}}
\newcommand{\gH}{\mathfrak{H}}
\newcommand{\gI}{\mathfrak{I}}
\newcommand{\gJ}{\mathfrak{J}}
\newcommand{\gK}{\mathfrak{K}}
\newcommand{\gL}{\mathfrak{L}}
\newcommand{\gM}{\mathfrak{M}}
\newcommand{\gN}{\mathfrak{N}}
\newcommand{\gO}{\mathfrak{O}}
\newcommand{\gP}{\mathfrak{P}}
\newcommand{\gQ}{\mathfrak{Q}}
\newcommand{\gR}{\mathfrak{R}}
\newcommand{\gS}{\mathfrak{S}}
\newcommand{\gT}{\mathfrak{T}}
\newcommand{\gU}{\mathfrak{U}}
\newcommand{\gV}{\mathfrak{V}}
\newcommand{\gW}{\mathfrak{W}}
\newcommand{\gX}{\mathfrak{X}}
\newcommand{\gY}{\mathfrak{Y}}
\newcommand{\gZ}{\mathfrak{Z}}

%old calligraphic font
\newcommand{\calA}{\mathcal{A}}
\newcommand{\calB}{\mathcal{B}}
\newcommand{\calC}{\mathcal{C}}
\newcommand{\calD}{\mathcal{D}}
\newcommand{\calE}{\mathcal{E}}
\newcommand{\calF}{\mathcal{F}}
\newcommand{\calG}{\mathcal{G}}
\newcommand{\calH}{\mathcal{H}}
\newcommand{\calI}{\mathcal{I}}
\newcommand{\calJ}{\mathcal{J}}
\newcommand{\calK}{\mathcal{K}}
\newcommand{\calL}{\mathcal{L}}
\newcommand{\calM}{\mathcal{M}}
\newcommand{\calN}{\mathcal{N}}
\newcommand{\calO}{\mathcal{O}}
\newcommand{\calP}{\mathcal{P}}
\newcommand{\calQ}{\mathcal{Q}}
\newcommand{\calR}{\mathcal{R}}
\newcommand{\calS}{\mathcal{S}}
\newcommand{\calT}{\mathcal{T}}
\newcommand{\calU}{\mathcal{U}}
\newcommand{\calV}{\mathcal{V}}
\newcommand{\calW}{\mathcal{W}}
\newcommand{\calX}{\mathcal{X}}
\newcommand{\calY}{\mathcal{Y}}
\newcommand{\calZ}{\mathcal{Z}}

%sf font for transposition and spaces like R
\newcommand{\sfA}{\mathsf{A}}
\newcommand{\sfB}{\mathsf{B}}
\newcommand{\sfC}{\mathsf{C}}
\newcommand{\sfD}{\mathsf{D}}
\newcommand{\sfE}{\mathsf{E}}
\newcommand{\sfF}{\mathsf{F}}
\newcommand{\sfG}{\mathsf{G}}
\newcommand{\sfH}{\mathsf{H}}
\newcommand{\sfI}{\mathsf{I}}
\newcommand{\sfJ}{\mathsf{J}}
\newcommand{\sfK}{\mathsf{K}}
\newcommand{\sfL}{\mathsf{L}}
\newcommand{\sfM}{\mathsf{M}}
\newcommand{\sfN}{\mathsf{N}}
\newcommand{\sfO}{\mathsf{O}}
\newcommand{\sfP}{\mathsf{P}}
\newcommand{\sfQ}{\mathsf{Q}}
\newcommand{\sfR}{\mathsf{R}}
\newcommand{\sfS}{\mathsf{S}}
\newcommand{\sfT}{\mathsf{T}}
\newcommand{\sfU}{\mathsf{U}}
\newcommand{\sfV}{\mathsf{V}}
\newcommand{\sfW}{\mathsf{W}}
\newcommand{\sfX}{\mathsf{X}}
\newcommand{\sfY}{\mathsf{Y}}
\newcommand{\sfZ}{\mathsf{Z}}

%\sectionfont{\centering}

%\subsectionfont{\centering}
%\subsubsectionfont{\normalsize}
%\setcounter{page}{1}


\author{Звонарев Никита}
\title{Односторонние веса ряда}
\begin{document}

\section{Постановка задачи нахождения весов}
\subsection{Общие соображения}
В этом разделе рассмотрим различные способы нахождения матричных весов $c_i$, $i = 1, \ldots, K$ с целью получения приближённо равных весов ряда $q_i$, $i = 1, \ldots, N$. Теорема \footnote{ссылка на соотношение между весами} дает необходимые и достаточные условия существования точных единичных весов ряда. Однако, мы тут же сталкиваемся с двумя проблемами. Во-первых, $N$ должно быть кратно $L$. Чаще всего $L$ стоит брать приближенно равным $N/2$, но $N$ может быть как нечетным, так и вовсе простым числом! Естественно, как резкое ограничение на возможные длины окна $L$, так и возможное обрезание ряда $\tsX$ до кратной $L$ длины крайне нежелательны. Во-вторых, полученные в теореме (?) веса $c_i$ содержат нули, что неприменимо, согласно \footnote{замечание про кривой Cadzow(0)}. В алгоритме Cadzow($\alpha$) \footnote{ссылка на статью} предложено заменять нули на $\alpha$. Согласно \footnote{ссылка на статью}, параметр $\alpha$ напрямую влияет на скорость сходимости метода Cadzow($\alpha$): меньшие $\alpha$ замедляют сходимость метода.

Введём следующие ограничения на допустимые веса $c_i$: во-первых, все $c_i > 0$. Во-вторых, по аналогии с алгоритмом Cadzow($\alpha$), введём параметр $\alpha$, задающий скорость сходимости алгоритма. Так как при умножении весов $c_i$ на константу алгоритм Oblique Cadzow нисколько не меняется, естественно рассматривать следующее ограничение на веса: 
\begin{equation} \label{eq:ratiocond}
\frac{\min_i c_i}{\max_i c_i} \ge \alpha.
\end{equation}
Именно из-за вышеназванной особенности рассматривается такое, на первый взгляд, странное ограничение (вместо, например, $c_i > \alpha$, $i = 1, \ldots, K$). Данное предположение, как будет показано позже в Разделе \footnote{ссылка}, подтверждается на численных примерах.

\subsection{Общая постановка задачи аппроксимации весов}
Для удобства перепишем уравнение (ссылка на уравнение, связывающее $c_i$ и $q_i$) в матричном виде. Для этого рассмотрим матрицу $\bfT$ порядка $N \times K$, имеющую следующий вид:
\begin{equation} \label{eq:tmatrix}
\bfT = (t_{i, j}), \quad t_{i, j} = \begin{cases}
1, & \text{для} \; i = j, \ldots, j + L - 1, \\
0, & \text{в противном случае}.
\end{cases}
\end{equation}
Таким образом, вектор весов ряда $Q = (q_1, \ldots, q_N)^\rmT$ может быть выражен через вектор матричных весов $C = (c_1, \ldots, c_K)^\rmT$ как $Q = \bfT C$. Заметим, что матрица $\bfT$ является матрицей полного ранга.

В общем виде задача формулируется следующим образом:
\begin{multline} \label{eq:commonw}
\|\bfT C - \widetilde W\|_\cdot \to \min_{c_i, i = 1, \ldots, K} \quad \text{при условиях} \;
c_i > 0, \; i = 1, \ldots, K, \; 
\frac{\min_i c_i}{\max_i c_i} \ge \alpha,
\end{multline}
где $\widetilde W = (\tilde w_1, \ldots, \tilde w_N)$ --- требуемые веса ряда (в нашем случае мы берём $\widetilde W = I_N = (1, \ldots, 1)^\rmT$ --- вектор из $N$ единиц), $0 \le \alpha \le 1$ --- параметр, регулирующий сходимость алгоритма Oblique Cadzow, $\|\cdot\|_\cdot$ --- норма в $\sfR^N$. Рассмотрим следующие стандартные нормы:
\begin{enumerate}
	\item $\|X\|_\cdot = \|X\|_\infty = \max_i |x_i|$,
	\item $\|X\|_\cdot = \|X\|_1 = \sum_i |x_i|$ --- норма, порождающая манхэттенскую метрику,
	\item $\|X\|_\cdot = \|X\|_2 = \sqrt{\sum_i x_i^2}$ --- обычная евклидова норма.
\end{enumerate}
Рассмотрим каждый случай по отдельности.

\subsection{Случай нормы $\|X\|_\infty$}
Покажем, что данная задача сводится к задаче линейного программирования с линейными ограничениями с помощью добавления вспомогательных переменных.

Введём $c_\text{max}$ --- дополнительную переменную, хранящую максимальный вес, и перейдём к новым переменным $\hat c_i = c_\text{max} - c_i$, $i = 1, \ldots, K$ --- разницу между максимальным среди всех и текущим весом, при этом $c_\text{max} \ge 0$, $\hat c_i \ge 0$. Между векторами $C = (c_1, \ldots, c_K)^\rmT$ и $\widehat C = (\hat c_1, \ldots, \hat c_K, c_\text{max})^\rmT$ существует простое линейное преобразование: $C = \bfH \widehat C$,
\begin{equation} \label{eq:hmatrix}
\bfH = \left(
\begin{array}{c|c}
\raisebox{-15pt}{{\huge\mbox{{$-\bfI_K$}}}} &  1 \\[-4ex]
 & \vdots \\
 & 1
\end{array}
\right),
\end{equation}
где $\bfI_K$ --- единичная матрица порядка $K \times K$. Условие \eqref{eq:ratiocond}, устанавливающее границу снизу для весов, в новых обозначениях записывается как $(1 - \alpha) c_\text{max} - \hat c_i \ge 0$, $i = 1, \ldots, K$.

Рассмотрим $\omega$ --- еще один дополнительный параметр, который хранит значение целевой функции. Заметим, что при любых дополнительных условиях, любой матрице $\bfA \in \sfR^{N \times K}$, любых $Y \in \sfR_N$, $X \in \sfR_K$ следующие задачи эквивалентны: 
\begin{equation*}
\|\bfA X - Y \|_\infty \to \min_{X}
\end{equation*}
и 
\begin{gather*}
\omega \to \min_{X, \omega} \quad \text{при условиях} \\ \bfA X = F = (f_1, \ldots, f_N)^\rmT, \quad f_i - y_i \le \omega, \quad f_i - y_i \ge -\omega, \quad i = 1, \ldots, N. 
\end{gather*}

Таким образом, в терминах линейного программирования задача \eqref{eq:commonw} в случае нормы $\|X\|_\infty$ переписывается следующим образом:
\begin{gather*}
\omega \to \min_{c_\text{max}, \, \hat c_1, \ldots, \hat c_K, \, \omega} \quad \text{при условиях} \\ \bfT \bfH \widehat C = F = (f_1, \ldots, f_N)^\rmT, \quad f_i - 1 \le \omega, \quad f_i - 1 \ge -\omega, \quad i = 1, \ldots, N, \\
c_\text{max} \ge 0, \quad \hat c_j \ge 0, \quad (1 - \alpha) c_\text{max} - \hat c_j \ge 0, \quad j = 1, \ldots, K.
\end{gather*}

\subsection{Случай нормы $\|X\|_1$}
Аналогично предыдущему случаю, задача \eqref{eq:commonw} может быть записана в терминах линейного программирования.

Для этого заметим следующее: если ввести $2N$ дополнительных переменных \\ $\kappa_1, \ldots, \kappa_N$, $\theta_1, \ldots, \theta_N$, то при любых дополнительных условиях, любой матрице $\bfA \in \sfR^{N \times K}$, любых $Y \in \sfR_N$, $X \in \sfR_K$ следующие задачи эквивалентны: 
\begin{equation*}
\|\bfA X - Y \|_1 \to \min_{X}
\end{equation*}
и 
\begin{gather*}
\sum_{i = 1}^N (\kappa_i + \theta_i) \to \min_{X, \, \kappa_1, \ldots, \kappa_N, \, \theta_1, \ldots, \theta_N} \quad \text{при условиях} \\ \bfA X = F = (f_1, \ldots, f_N)^\rmT, \quad f_i - y_i = \kappa_i - \theta_i, \quad \kappa_i \ge 0, \quad \theta_i \ge 0, \quad i = 1, \ldots, N. 
\end{gather*}

В итоге, задача \eqref{eq:commonw} в терминах линейного программирования в случае нормы $\|X\|_1$ переписывается следующим образом:
\begin{gather*}
\sum_{i = 1}^N (\kappa_i + \theta_i) \to \min_{c_\text{max}, \, \hat c_1, \ldots, \hat c_K, \, \kappa_1, \ldots, \kappa_N, \, \theta_1, \ldots, \theta_N} \quad \text{при условиях} \\ \bfT \bfH \widehat C = F = (f_1, \ldots, f_N)^\rmT,  \quad f_i - 1 = \kappa_i - \theta_i, \quad \kappa_i \ge 0, \quad \theta_i \ge 0, \quad i = 1, \ldots, N, \\
c_\text{max} \ge 0, \quad \hat c_j \ge 0, \quad (1 - \alpha) c_\text{max} - \hat c_j \ge 0, \quad j = 1, \ldots, K.
\end{gather*}

\end{document}
